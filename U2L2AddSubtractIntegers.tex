
\section{Add and Subtract Integers}

In this lesson, we will learn how to add and subtract positive and negative integers.

\subsection{Add Integers}
I will show two methods to add integers. You choose the method which makes 
more sense for you. We will calculate $(-1)+(-2)$.
\begin{method}
	\item The first method is the traditional number line method. Recall that the 
right side is the positive direction, and the left side is negative direction.
	\begin{steps}
		\item Locate the first number, $-1$, on the number line.
		\begin{tightcenter}
			\begin{tikzpicture}
				\begin{axis}[
						xmin=-5,xmax=5,
						ymin=-1,ymax=2,
						axis y line=none,
						height =1cm,
						grid=none,
						xtick={-4,-3,...,4},
						xlabel={}
					]
					\addplot+[soldot]coordinates{ (-1,0) };				
				\end{axis}
			\end{tikzpicture}
			\captionof{figure}{Locate $-1$ on the number line}
		\end{tightcenter}
		\item The second number is $-2$, meaning we will move to the \emph{left} (negative direction) by two units.
		\begin{tightcenter}
			\begin{tikzpicture}
				\begin{axis}[
						xmin=-5,xmax=5,
						ymin=-1,ymax=2,
						axis y line=none,
						height =1cm,
						grid=none,
						xtick={-4,-3,...,4},
						xlabel={}
					]
					\addplot[soldot]coordinates{ (-1,0) };	
					\addplot[<-,line width=3pt,red,domain=-3:-1]{0.5} node[pos=0.5,anchor=south]{$-2$};			
				\end{axis}
			\end{tikzpicture}
			\captionof{figure}{From $-1$ move to the left by $2$ units}
		\end{tightcenter}
		\item After the move, we reached the number $-3$ on the number line. This implies that
		\[
			(-1)+(-2)=-3
		\]
	\end{steps}
	\item The second method uses a money model. We deal with money every day, so most students easily understand this method.
	
	\begin{steps}
	\item Say you are gambling. The first number is $-1$, meaning you lost \$1 in the first game. 
	\item The second number is $-2$, meaning you lost \$2 in the second game. 
	\item Since you lost in both games, altogether, you lost. This implies the answer must be negative.
	\item Since you lost in both games, all together, you lost \$1+\$2=\$3.
	\end{steps}
	Finally, we have \[(-1)+(-2)=-3\]
\end{method}

Let's look at a few more examples.

\begin{myexample}
Calculate $4+(-5)$
\end{myexample}
\begin{solution}
	We will use the number line method.
	\begin{steps}
		\item Locate the first number, $4$, on the number line.
		\begin{tightcenter}
			\begin{tikzpicture}
				\begin{axis}[
						xmin=-5,xmax=5,
						ymin=-1,ymax=1,
						axis y line=none,
						height =1cm,
						grid=none,
						xtick={-4,-3,...,4},
						xlabel={}
					]
					\addplot+[soldot]coordinates{ (4,0) };				
				\end{axis}
			\end{tikzpicture}
			\captionof{figure}{Locate $4$ on the number line}
		\end{tightcenter}
		\item The second number is $-5$, meaning we will move to the \emph{left} (negative direction) by five units.
		\begin{tightcenter}
			\begin{tikzpicture}
				\begin{axis}[
						xmin=-5,xmax=5,
						ymin=-1,ymax=2,
						axis y line=none,
						height =1cm,
						grid=none,
						xtick={-4,-3,...,4},
						xlabel={}
					]
					\addplot[<-,line width=3pt,red,domain=-1:4]{0.5} node[pos=0.5,anchor=south]{$-5$};	
					\addplot[soldot]coordinates{ (4,0) };		
				\end{axis}
			\end{tikzpicture}
			\captionof{figure}{From $4$ move to the left by $5$ units}
		\end{tightcenter}
		\item After the move, we reached the number $-1$ on the number line. This implies:
		\[
			4+(-5)=-1
		\]
	\end{steps}
\end{solution}

\begin{myexample}
Calculate $(-4)+5$
\label{ex:e-4+5}
\end{myexample}
\begin{solution}
We will use the money model to solve this problem.
	\begin{steps}
	\item Say you are gambling. The first number is $-4$, meaning you lost \$4 in the first game. 
	\item The second number is $5$, meaning you won \$5 in the second game. 
	\item Since you won more money than you lost, altogether, you won. This implies the answer must be positive.
	\item Since you lost some and then won some, we should find the difference of those two numbers' absolute values: \$$5-$\$$4$=\$$1$.
	\end{steps}
	Finally, we have \[(-4)+5=1\]
\end{solution}

If you are new to negative numbers, the number line method can help you understand integer operations. When numbers are big, it's difficult to locate numbers on the number line; the money model would work better.

\begin{myexample}
Calculate $(-44)+15$
\end{myexample}
\begin{solution}
We will use the money model to solve this problem.
	\begin{steps}
	\item Say you are gambling. The first number is $-44$, meaning you lost \$44 in the first game. 
	\item The second number is $15$, meaning you won \$15 in the second game. 
	\item Since you lost more money than you won, altogether, you lost. This implies the answer must be negative.
	\item Since you lost some and then won some, we should find the difference of those two numbers' absolute values: \$$44-$\$$15$=\$$29$.
	\end{steps}
	Finally, we have \[(-44)+15=-29\]
	Don't forget the negative sign in the answer (because you lost money).
\end{solution}

\subsection{Subtract a Positive Integer}
Let's observe a pattern first:
\begin{align*}
	&3-2 = 1 \longleftrightarrow 3+(-2) =1 \\
	&4-3 =1  \longleftrightarrow 4+(-3) =1 \\
	&1-3 =-2  \longleftrightarrow 1+(-3) =-2 
\end{align*}

We can see the subtraction sign and negative symbol have the same functions! Starting today, it would be great if you can treat the subtraction sign as a negative symbol. When we subtract a positive number, we can treat it as "adding a negative number", and then use the methods we learned earlier to add integers.

\begin{myexample}
Calculate $-4-5$
\end{myexample}
\begin{solution}
First, we change subtraction to "adding a negative":
\[ -4-5=(-4)+(-5) \]
Note that the parentheses around $-4$ is optional, just to make it clear. The parentheses around $-5$ is needed, as it's confusing to write two symbols right next to each other, like $-4+-5$.
Next, we will use the money model to solve this problem.
	\begin{steps}
	\item Say you are gambling. The first number is $-4$, meaning you lost \$4 in the first game. 
	\item The second number is $-5$, meaning you lost \$5 in the second game. 
	\item Since you lost money in both games, altogether, you lost. This implies the answer must be negative.
	\item Since you lost in both games, we should find the sum of those two numbers' absolute values: \$$5+$\$$4$=\$$9$.
	\end{steps}
	Finally, we have \[-4-5=(-4)+(-5)=-9\]
\end{solution}

\begin{myexample}
Calculate $4-5$
\end{myexample}
\begin{solution}
First, we change subtraction to "adding a negative":
\[ 4-5=4+(-5) \]
Next, we will use the money model to solve this problem.
	\begin{steps}
	\item Say you are gambling. The first number is $4$, meaning you won \$4 in the first game. 
	\item The second number is $-5$, meaning you lost \$5 in the second game. 
	\item Since you lost more money than you won, altogether, you lost. This implies the answer must be negative.
	\item Since you won some money and then lost some, we should find the difference of those two numbers' absolute values: \$$5-$\$$4$=\$$1$.
	\end{steps}
	Finally, we have \[4-5=4+(-5)=-1\]
\end{solution}

Don't be silly when you do problems like $9-5$. There is no need to use the number line or money model, as $9-5=4$. :)

\subsection{Subtract a Negative Integer}
Here is the bottom line: When two negative signs are right next to each other, we change these two negative signs to one positive sign, as in
\[ 1-(-2)=1+2 \]

For now, memorize this as a rule. In the next lesson, we will understand why.

Look at the difference between these two problems:
\[
\begin{aligned}[t]
   &\phantom{{}=}1-(-2) &\phantom{aaaaaaaaaaaaaaaaaaa}& \phantom{{}=}-1-2 \\
   &=1+2 &\phantom{aaaaaaaaaaaaaaaaaaa} & =-3 \\
   &=3
\end{aligned}
\]

We only change two negative signs to one positive sign if they are right next to each other. In $-1-2$, a number separated those two negative signs, so we cannot change them to one positive sign.

\begin{myexample}
Calculate $-4-(-5)$
\end{myexample}
\begin{solution}
	First, we change two negative signs to one positive sign:
	\[ -4-(-5)=-4+5 \]
	The rest of the solution is the same as in \cref{ex:e-4+5}.
	Finally, we have \[-4-(-5)=-4+5=1\]
\end{solution}

Finally, let's look at an example to put together what we learned in this lesson.

\begin{myexample}
Calculate $-3-(-5)-(-9)-14$
\begin{solution}
	The first step is to change two negative signs (right next to each other) to one positive sign:
		\[
		\begin{aligned}[t]
			&\phantom{{}=} -3-(-5)-(-9)-14 \\
			& = -3+5+9-14 \\
		\end{aligned}
		\]
	Next, we use either the number line model or money model to do additions and subtractions step by step. The full solution is:
		\[
		\begin{aligned}[t]
			&\phantom{{}=} -3-(-5)-(-9)-14 \\
			& = -3+5+9-14 \\
            & = 2+9-14     \\
            & = 11-14 \\
            &= -3
		\end{aligned}
		\]
\end{solution}
\end{myexample}


