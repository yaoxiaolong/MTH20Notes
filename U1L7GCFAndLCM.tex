\section{Greatest Common Factor and Least Common Multiple}

In this lesson, we learn how to find the GCF (Greatest Common Factor) and LCM (Least Common Multiple) of a group of numbers (usually two numbers).

For example, the GCF of $6$ and $8$ is $2$. The LCM of $6$ and $8$ is $24$. These two concepts will help you better understand numbers. In later lessons, we need to find LCM of denominators when we add or subtract fractions.

\subsection{GCF (Greatest Common Factor)}
We will find the GCF of $12$ and $18$. We will list both numbers' factors:

\begin{itemize}
\item Factors of $12$ are: $1,2,3,4,6,12$.
\item Factors of $18$ are: $1,2,3,6,9,18$.
\end{itemize}

The \textit{common} factors of $12$ and $18$ are: $1,2,3,6$.

The GCF (Greatest Common Factor) of $12$ and $18$ is $6$.

Note that we listed all factors of $12$ and $18$ in order to find their GCF. There is an easier method. We will prime factor both numbers:
\[
\begin{tabular}{ l c r }
  \begin{forest} [12[2][6[2][3]]]\end{forest} &  & \begin{forest} [18[2][9[3][3]]]\end{forest} \\
\end{tabular}
\]
We have:
\[
\begin{aligned}[t]
&12=2\cdot2\cdot3 \\
&18=2\cdot3\cdot3
\end{aligned}
\]

Since $12$ and $18$ share the following factors, one $2$ and one $3$, the GCF of $12$ and $18$ is $2\cdot3=6$.

Let's look at another example.

\begin{myexample}
Find the GCF of $24$ and $36$.
\end{myexample}
\begin{solution}
We will prime factor both $24$ and $36$:
\[
\begin{tabular}{ l c r }
  \begin{forest} [24[2][12[2][6[2][3]]]]\end{forest} &  & \begin{forest} [36[2][18[2][9[3][3]]]]\end{forest} \\
\end{tabular}
\]
We have:
\[
\begin{aligned}[t]
&24=2\cdot2\cdot2\cdot3 \\
&36=2\cdot2\cdot3\cdot3
\end{aligned}
\]
We can see $24$ and $36$ share the following factors, two $2$'s and one $3$, so the GCF of $24$ and $36$ is $2\cdot2\cdot3=12$.
\end{solution}

\begin{myexample}
Find the GCF of $8$ and $15$.
\end{myexample}
\begin{solution}
Prime factor both numbers, we have:
\[
\begin{aligned}[t]
&8=2\cdot2\cdot2 \\
&15=3\cdot5
\end{aligned}
\]
We can see $8$ and $15$ don't share any factors. The GCF of $8$ and $15$ is $1$, because $1$ is a factor of all natural numbers.
\end{solution}

\begin{myexample}
Find the GCF of $5$ and $7$.
\end{myexample}
\begin{solution}
Both $5$ and $7$ are prime numbers, so it's impossible for them to share any factor except $1$. So $1$ is the GCF of $5$ and $7$.
\end{solution}

\begin{myexample}
Find the GCF of $10$ and $20$.
\end{myexample}
\begin{solution}
Since $10$ goes into $20$ evenly, the GCF of $10$ and $20$ is simply $10$, because $10$ goes into both $10$ and $20$ evenly.
\end{solution}

\subsection{LCM (Least Common Multiple)}
We will find the LCM of $6$ and $8$. First, we will find the first few multiples of $6$ and $8$:
\begin{itemize}
\item The first few multiples of $6$ are: $6,12,18,24,30,36,42,48,54,...$.
\item The first few multiples of $8$ are: $8,16,24,32,40,48,56...$.
\end{itemize}

We can see $6$ and $8$ have some \textit{common} multiples: $24,48,...$. The \textit{least} common multiple is $24$.

Later, when we do $\frac{1}{6}+\frac{1}{8}$, we need to change the denominator of both fractions to $24$, the LCM of $6$ and $8$. We will learn details later, but here is the solution:
\[
\begin{aligned}[t]
   &\phantom{{}=}\frac{1}{6}+\frac{1}{8} \\
   &= \frac{1\cdot4}{6\cdot4}+\frac{1\cdot3}{8\cdot3} \\
   &= \frac{4}{24}+\frac{3}{24} \\
   &= \frac{4+3}{24} \\
   &= \frac{7}{24}
\end{aligned}
\]

As an example, we will find the LCM of $6$ and $8$ in two methods.
\subsubsection{Method 1 to Find LCM}
We will prime factor both $6$ and $8$:
\[
\begin{aligned}[t]
&6=2\cdot3 \\
&8=2\cdot2\cdot2
\end{aligned}
\]

The LCM of $6$ and $8$ must \textit{cover} the prime factors of both numbers:
\begin{itemize}
\item The LCM of $6$ and $8$ must have three $2$'s as factors, as $6$ needs one $2$ and $8$ needs three $2$'s.
\item The LCM of $6$ and $8$ must have one $3$ as a factor, as $6$ needs one $3$.
\end{itemize}
Putting them together, the LCM must have three $2$'s and one $3$ as factors. We have: $\text{LCM}=2\cdot2\cdot2\cdot3=24$.

To verify, we have $\frac{24}{6}=4$, and $\frac{24}{8}=3$.

\subsubsection{Method 2 to Find LCM}
List the first few multiples of the \textit{bigger} number until the smaller number goes into a multiple evenly:

The first few multiples of $8$ are: $8,16,24$.

We stopped here because $6$ goes into $24$ evenly, so $24$ is the LCM of $6$ and $8$. To verify, we have $\frac{24}{6}=4$, and $\frac{24}{8}=3$.

Which method should we use? Most of the time, the second method is faster, so try the second method first. Sometimes the first method is faster.

\begin{myexample}
Find the LCM of $4$ and $12$.
\end{myexample}
\begin{solution}
Since $4$ goes into $12$, the LCM of $4$ and $12$ is simply $12$.

To verify, we have $\frac{12}{4}=3$ and $\frac{12}{12}=1$.
\end{solution}

\begin{myexample}
Find the LCM of $8$ and $18$.
\end{myexample}
\begin{solution}
We list the first few multiples of $18$, until $8$ goes evenly into one of them:

Multiples of $18$: $18$,$36$,$54$,$72$.

We stop here because $\frac{72}{8}=9$. The LCM of $8$ and $18$ is $72$.

To verify, we have $\frac{72}{8}=9$ and $\frac{72}{18}=4$.
\end{solution}

\begin{myexample}
Find the LCM of $24$ and $28$.
\end{myexample}
\begin{solution}
We list the first few multiples of $28$, until $24$ goes evenly into one of them:

Multiples of $28$: $28$,$56$,$84$,$112$,$140$...

$24$ doesn't go evenly into any of them. After listing $5$ multiples, we know this method is not the best method to find the LCM. We will try the other method. We will prime factor both numbers:
\[
\begin{tabular}{ l c r }
  \begin{forest} [28[2][14[2][7]]]\end{forest} &  & \begin{forest} [24[2][12[2][6[2][3]]]]\end{forest} \\
\end{tabular}
\]
We have:
\[
\begin{aligned}[t]
&28=2\cdot2\cdot7 \\
&24=2\cdot2\cdot2\cdot3
\end{aligned}
\]
The LCM of $28$ and $24$ must \textit{cover} the prime factors of both numbers:
\begin{itemize}
\item The LCM must have three $2$'s as factors, as $24$ needs three $2$'s and $28$ needs two $2$'s.
\item The LCM must have one $3$ as a factor, as $24$ needs one $3$.
\item The LCM must have one $7$ as a factor, as $28$ needs one $7$.
\end{itemize}
Putting them together, the LCM must have three $2$'s, one $3$ and one $7$ as factors. We have: $\text{LCM}=2\cdot2\cdot2\cdot3\cdot7=168$.

To verify, we have $\frac{168}{28}=6$, and $\frac{168}{24}=7$.
\end{solution}
