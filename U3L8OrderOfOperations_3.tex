
\section{Order of Operations involving Fractions}

In this lesson, we will do a few order-of-operations problems involving fractions.

We have learned the order of operations in earlier lessons:

\[
\begin{aligned}[t]
   &P &\text{(Parentheses)} \\
   &E &\text{(Exponent)} \\
   &MD &\text{(Multiplication and Division)} \\
   &AS &\text{(Addition and Subtraction)}
\end{aligned}
\]
\captionof{figure}{Order of Operations}
\label{fig:PEMDAS38}

In this lesson, the rules didn't change, except fractions are involved. We will simply look at a few examples.

\begin{myexample}
\[
\begin{aligned}[t]
	&\phantom{{}=} \frac{5}{4}+(\frac{3}{4})2 \\
	&= \frac{5}{4}+\frac{3}{4} \cdot \frac{2}{1} \\
	&= \frac{5}{4}+\frac{3}{4\div2} \cdot \frac{2\div2}{1} \\
	&= \frac{5}{4}+\frac{3}{2} \cdot \frac{1}{1} \\
   	&= \frac{5}{4}+\frac{3}{2} \\
   	&= \frac{5}{4}+\frac{3\cdot2}{2\cdot2} \\
   	&= \frac{5}{4}+\frac{6}{4} \\
	&= \frac{11}{4}
\end{aligned}
\]
Without specified instruction, there is no need to change $\frac{11}{4}$ into $3\frac{3}{4}$.
\end{myexample}

\begin{myexample}
\[
\begin{aligned}[t]
	&\phantom{{}=} \frac{4}{3}-5(\frac{1}{9}-\frac{1}{6}) \\
	&= \frac{4}{3}-5(\frac{1\cdot2}{9\cdot2}-\frac{1\cdot3}{6\cdot3}) \\
	&= \frac{4}{3}-5(\frac{2}{18}-\frac{3}{18}) \\
	&= \frac{4}{3}-5(-\frac{1}{18}) \\
	&= \frac{4}{3}+5(\frac{1}{18}) &(\text{negative})\cdot(\text{negative})=\text{positive}\\
	&= \frac{4}{3}+\frac{5}{1} \cdot \frac{1}{18} \\
	&= \frac{4}{3}+\frac{5}{18} \\
	&= \frac{4\cdot6}{3\cdot6}+\frac{5}{18} \\
	&= \frac{24}{18}+\frac{5}{18} \\
	&= \frac{29}{18}
\end{aligned}
\]
In the step 
\[ \frac{4}{3}-5(-\frac{1}{18}) \] 
we treat the subtraction symbol as a negative symbol, so we have 
\[ \frac{4}{3}+(-5)\cdot(-\frac{1}{18}) \]
Since $(\text{negative})\cdot(\text{negative})=\text{positive}$, we have 
\[ \frac{4}{3}+(-5)\cdot(-\frac{1}{18})=\frac{4}{3}+5(\frac{1}{18}) \]
\end{myexample}

\begin{myexample}
Compare the following example with the last one, and see the difference between parentheses and absolute value.

\[
\begin{aligned}[t]
	&\phantom{{}=} \frac{4}{3}-5 \left| \frac{1}{9}-\frac{1}{6} \right| \\
	&= \frac{4}{3}-5 \left| \frac{1\cdot2}{9\cdot2}-\frac{1\cdot3}{6\cdot3} \right| \\
	&= \frac{4}{3}-5 \left| \frac{2}{18}-\frac{3}{18} \right| \\
	&= \frac{4}{3}-5 \left| -\frac{1}{18} \right| \\
	&= \frac{4}{3}-5(\frac{1}{18}) \\
	&= \frac{4}{3}-\frac{5}{1} \cdot \frac{1}{18} \\
	&= \frac{4}{3}-\frac{5}{18} \\
	&= \frac{4\cdot6}{3\cdot6}-\frac{5}{18} \\
	&= \frac{24}{18}-\frac{5}{18} \\
	&= \frac{19}{18}
\end{aligned}
\]
\end{myexample}

\begin{myexample}
Compare these two examples:

\begin{tabular}[t]{c@{\hspace{4cm}}c@{\hspace{2cm}}c}
&
$ \begin{aligned}[t] 
	&\phantom{{}=} 1-(\frac{2}{3})^{2} \\ 
	&= 1-(\frac{2}{3})(\frac{2}{3}) \\ 
	&= 1-\frac{2\cdot2}{3\cdot3} \\
	&= 1-\frac{4}{9} \\
	&= \frac{9}{9}-\frac{4}{9} \\
	&= \frac{5}{9}
  \end{aligned} $ 
&
$ \begin{aligned}[t] 
	&\phantom{{}=} 1-(-\frac{2}{3})^{2} \\ 
	&= 1-(-\frac{2}{3})(-\frac{2}{3}) \\ 
	&= 1-(\frac{2}{3})(\frac{2}{3}) \\ 
	&= 1-\frac{2\cdot2}{3\cdot3} \\
	&= 1-\frac{4}{9} \\
	&= \frac{9}{9}-\frac{4}{9} \\
	&= \frac{5}{9}
  \end{aligned} $ 
\end{tabular}

Note that $(\frac{2}{3})^{2}=\frac{4}{9}$, and $(-\frac{2}{3})^{2}=\frac{4}{9}$.
\end{myexample}

\begin{myexample}
Compare these two examples:

\begin{tabular}[t]{c@{\hspace{4cm}}c@{\hspace{2cm}}c}
&
$ \begin{aligned}[t] 
	&\phantom{{}=} 1-(\frac{2}{3})^{3} \\ 
	&= 1-(\frac{2}{3})(\frac{2}{3})(\frac{2}{3}) \\ 
	&= 1-\frac{2\cdot2\cdot2}{3\cdot3\cdot3} \\
	&= 1-\frac{8}{27} \\
	&= \frac{27}{27}-\frac{8}{27} \\
	&= \frac{19}{27}
  \end{aligned} $ 
&
$ \begin{aligned}[t] 
	&\phantom{{}=} 1-(-\frac{2}{3})^{3} \\ 
	&= 1-(-\frac{2}{3})(-\frac{2}{3})(-\frac{2}{3}) \\ 
	&= 1+(\frac{2}{3})(\frac{2}{3})(\frac{2}{3}) \\ 
	&= 1+\frac{2\cdot2\cdot2}{3\cdot3\cdot3} \\
	&= 1+\frac{8}{27} \\
	&= \frac{27}{27}+\frac{8}{27} \\
	&= \frac{35}{27}
  \end{aligned} $ 
\end{tabular}

Note that $1-(-\frac{2}{3})(-\frac{2}{3})(-\frac{2}{3})$ becomes $1+(\frac{2}{3})(\frac{2}{3})(\frac{2}{3})$ because each pair of negative symbols canceled each other.

Compare $(\frac{2}{3})^{3}=\frac{8}{27}$ and $(-\frac{2}{3})^{3}=-\frac{8}{27}$.

Earlier, we learned: $(\frac{2}{3})^{2}=\frac{4}{9}$, and $(-\frac{2}{3})^{2}=\frac{4}{9}$.

Instead of trying to memorize these results, understand that each pair of negative symbols cancel each other (in multiplication).
\end{myexample}


