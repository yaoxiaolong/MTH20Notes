
\section{Percent Word Problems}

We often encounter the following problems in real life:

\begin{itemize}
\item You make \$$15.00$ an hour, and you just got a $5\%$ raise! How much do you make now?
\item Jose's dinner bill was \$$90.00$, and he added \$$13.50$ as a tip. How much percent of a tip did Jose give?
\item Each vitamin supplement pill contains $54$mg of Vitamin C, which is $60\%$ of an adult's recommended daily Vitamin C intake. How many mg of Vitamin C should an adult consume every day?
\end{itemize}

After this lesson, you will be able to answer these questions.

In the last lesson, we learned three types of percent problems:
\begin{itemize}
\item \textbf{Type I:} What is $40\%$ of $20$?
\item \textbf{Type II:} $8$ is what percent of $20$?
\item \textbf{Type III:} $8$ is $40\%$ of what?
\end{itemize}

Each word problem can be boiled down to one of these three types.

\subsection{Type I Percent Word Problems}

\begin{myexample}
You make \$$15.00$ an hour, and you just got a $5\%$ raise! How much do you make now?
\end{myexample}
\begin{solution}
We will use the Percent Formula to solve this problem.

\textbf{Method 1: } This problem boils down to this question: What is $5\%$ of $15$? This is Type I percent problem. Assume $x$ is $5\%$ of $15$. We will write down the "Percent Formula" and the problem right next to each other:
\[
\begin{aligned}[t]
	&3 &= &&50\% &&\cdot &&6 \\
	&x &= &&5\% &&\cdot &&15
\end{aligned}
\]

Next, we can solve for $x$ in the equation. In this type of problems, $x$ happens to be alone on one side of the equal sign, so all we need to do is to do the calculation on the other side of the equal sign. Remember that $5\%=0.05$. We have:
\[
\begin{aligned}[t]
	x &= 5\% \cdot 15 \\
	x &= 0.05 \cdot 15 \\
	x &= 0.75
\end{aligned}
\]
Since $5\%$ of \$$15.00$ is \$$0.75$, after the raise, you are paid \$$15.00+$\$$0.75=$\$$15.75$ per hour.

\textbf{Method 2: } After a $5\%$ raise, your new pay is $105\%$ of your old pay. Now this problem boils down to this question: What is $105\%$ of $15$? This is Type I percent problem. Assume $x$ is $105\%$ of $15$. We can solve this equation:
\[
\begin{aligned}[t]
	x &= 105\% \cdot 15 \\
	x &= 1.05 \cdot 15 \\
	x &= 15.75
\end{aligned}
\]
\textbf{Conclusion:} After the raise, you are paid \$$15.75$ per hour.

\end{solution}

\subsection{Type II Percent Word Problem}
\begin{myexample}
Jose's dinner bill was \$$90.00$, and he added \$$13.50$ as a tip. How much percent of a tip did Jose give?
\end{myexample}
\begin{solution}
We need to find \$$13.50$ is what percent of the \$$90.00$. This is a Type II percent problem. We will use proportion to solve this problem.

Assume $13.5$ is $x\%$ of $90$. We can rephrase this sentence as: $13.5$ out of $90$ is like $x$ out of $100$. Here is the key: The number following the word "of" corresponds to $100$. Now we can set up and solve a proportion:
\[
\begin{aligned}[t]
	\frac{13.5}{90} &= \frac{x}{100} \\
	90x &= 13.5 \cdot 100 \\
	90x &= 1350 \\
	\frac{90x}{90} &= \frac{1350}{90} \\
	x &= 15
\end{aligned}
\]
\textbf{Conclusion:} Jose gave $15\%$ tip.

\end{solution}

\subsection{Type III Percent Problem}
\begin{myexample}
Each vitamin supplement pill contains $54$mg of Vitamin C, which is $60\%$ of an adult's recommended daily Vitamin C intake. How many mg of Vitamin C should an adult consume every day?
\end{myexample}
\begin{solution}

If an adult takes one pill, he/she has taken $54$mg of Vitamin C, or $60\%$ of a day's recommended Vitamin C intake. This problem can be boiled down to this problem: $54$ is $60\%$ of what? This is a Type III percent problem.

We will use multiplication/division to solve this problem. No variable ($x$) is involved in this method. The key is to write down a simply example on scratch paper, and then put numbers in their corresponding places.

To find "$3$ is $50\%$ of what", we do:
\[ 3\div0.5=6 \]
Similarly, to find "$54$ is $60\%$ of what", we do:
\[ 54\div0.6=90 \]
\textbf{Conclusion:} An adult should consume $90$ mg of Vitamin C every day.

\end{solution}


\begin{myexample}
Omar sells cars for a living. His monthly base pay is \$$1,200.00$. On top of his base pay, Omar keeps $2.5\%$ of his sales as commission. In a certain month, Omar took home a total of \$$2,950.00$. How much in car sales did Omar make in that month?
\end{myexample}
\begin{solution}
Omar took home a total of \$$2,950.00$. Out of this amount, \$$1,200.00$ was his base pay, so his commission pay was \$$2,950.00-$\$$1,200.00=$\$$1,750.00$.

The \$$1,750.00$ was his commission, which was $2.5\%$ of his sales in that month. Now this problem can be rephrased as: $1750$ is $2.5\%$ of what? This is a Type III percent problem. We will use the Percent Formula to solve this problem.

Assume $1750$ is $2.5\%$ of $x$. We will write down the "Percent Formula" and the problem right next to each other:
\[
\begin{aligned}[t]
	&3 &= &&50\% &&\cdot &&6 \\
	&1750 &= &&2.5\% &&\cdot &&x
\end{aligned}
\]

Next, we can solve for $x$ in the equation:
\[
\begin{aligned}[t]
	1750 &= 2.5\% \cdot x \\
	1750 &= 0.025x \\
	\frac{1750}{0.025} &= \frac{0.025x}{0.025} \\
	70000 &= x
\end{aligned}
\]
\textbf{Conclusion:} Omar made \$$70,000.00$ in car sales in that month.

\end{solution}


\subsection{Rounding}
Sometimes we need to round numbers, like in the next example.
\begin{myexample}
A small town has $981$ residents. Out of these residents, there are $45$ Asians. What percent of the town's population are Asians? Round your answer to the hundredths place.
\end{myexample}
\begin{solution}

This problem can be boiled down to this problem: $45$ is what percent of $981$? This is a Type II percent problem. We will use the Percent Formula to solve this problem. Assume $45$ is $x$ (as a percent) of $981$.

We will write down the "Percent Formula" and the problem right next to each other:
\[
\begin{aligned}[t]
	&3 &= &&50\% &&\cdot &&6 \\
	&45 &= &&x \text{ (as a percent)} &&\cdot &&981
\end{aligned}
\]

Next, we can solve for $x$ in the equation:
\[
\begin{aligned}[t]
	45 &= x \cdot 981 \\
	45 &= 981x \\
	\frac{45}{981} &= \frac{981x}{981} \\
	0.0459 &\approx x \\
	4.59\% &\approx x
\end{aligned}
\]
\textbf{Conclusion:} Approximately $4.59\%$ of the town's population are Asians.

\end{solution}

\subsection{Percent Problems involving Subtraction}
In percent problems, keep in mind that the whole is always $100\%$. See the next example.

\begin{myexample}
In a county, $36.2\%$ of registered voters are Democrats, $40.9\%$ of registered voters are Republicans. The rest of registered voters are Independents. The county has a total of $12,000$ registered voters. How many registered Independents live in this county?
\end{myexample}
\begin{solution}
Out of the county's $100\%$ registered voters, $36.2\%$ are Democrats, and $40.9\%$ are Republicans. This implies Independents make up $100\%-36.2\%-40.9\%=22.9\%$ of all registered voters. Now the problem becomes: What is $22.9\%$ of $12,000$?

We will use multiplication/division to solve this problem. No variable ($x$) is involved in this method. The key is to write down a simple example on scratch paper, and then put numbers in their corresponding places.

To find "$50\%$ of $6$", we do:
\[ 50\% \cdot 6=0.5\cdot6=3 \]
Similarly, to find "$22.9\%$ of $12,000$", we do:
\[ 22.9\% \cdot 12400=0.229\cdot12000= 2748\]
\textbf{Conclusion:} The county has $2,748$ registered Independent voters.

\end{solution}

\subsection{More than $100\%$}
Once you are done with a percent problem, quickly use mental math to check your solution and see whether it makes common sense.

\begin{myexample}
In a high school, the number of African American students is $120\%$ of the number of Asian students. If there are $72$ African American students, how many Asian students are there?
\end{myexample}
\begin{solution}
It's given that the number of African American students is $120\%$ of the number of Asian students. This implies there are more African American students than Asian students in this school. We will use this to quickly check our answer later.

We will use proportion to solve this problem. Assume the school has $x$ Asian students. This problem boils down to this question: $72$ is $120\%$ of what? We can rephrase this sentence as: $72$ out of $x$ is like $120$ out of $100$. Here is the key: The number following the word "of" corresponds to $100$. Now we can set up and solve a proportion:
\[
\begin{aligned}[t]
	\frac{72}{x} &= \frac{120}{100} \\
	120x &= 72 \cdot 100 \\
	120x &= 7200 \\
	\frac{120x}{120} &= \frac{7200}{120} \\
	x &= 60
\end{aligned}
\]
\textbf{Conclusion:} The school has $60$ Asian students.

Our solution shows there are more African American students than Asian students in this school, which matches the given information we mentioned earlier. If the result shows more Asian students, we would know we made a mistake.
\end{solution}


