
\section{Unit Conversion}

In the US, the American measurement system is used. In most other countries, the metric system is used. In this lesson, we will learn how to convert measurements within either the American or the metric system.

\subsection{American System Conversions}
We will start with yard, feet and inches. Everyone should memorize the following relationships:
\begin{itemize}
\item $1$ yard = $3$ feet
\item $1$ foot = $12$ inches
\end{itemize}

There are three methods to do conversions. We will look at an example.

\begin{myexample}
Convert $12.3$ feet to yards.
\end{myexample}
\begin{solution}
\textbf{Method 1:} We will use proportion to solve this problem. Assume $12.3$ feet is equivalent to $x$ yards. First, we organize given information in a table:
\begin{center}
\begin{tabular}{ | c | c | c | }
	\hline
    	& \textbf{Rate} & \textbf{Conversion} \\ \hline
  \textbf{feet} & $3$ & $12.3$ \\ \hline
  \textbf{yard} & $1$ & $x$ \\ \hline
\end{tabular}
\end{center}

Now we can set up the proportion and then solve for $x$:
\[
\begin{aligned}[t] 
	\frac{3 \text{ feet}}{1 \text{ yard}} &= \frac{12.3 \text{ feet}}{x \text{ yards}} \\
	3x &= 1\cdot12.3 \\ 
	3x &= 12.3 \\
	\frac{3x}{3} &= \frac{12.3}{3} \\ 
	x &= 4.1
\end{aligned}
\]

\textbf{Conclusion:} $12.3$ feet is equivalent to $4.1$ yards.

\textbf{Method 2:} We will use fraction multiplication to solve this problem. The rate can be written as either $\frac{3 \text{ feet}}{1 \text{ yard}}$ or $\frac{1 \text{ yard}}{3 \text{ feet}}$. Since the given number is $12.3$ feet, we will use the rate $\frac{1 \text{ yard}}{3 \text{ feet}}$, so the unit "feet" will cancel:

\[
\begin{aligned}[t]
	&\phantom{{}=}12.3 \text{ feet} \cdot \frac{1 \text{ yard}}{3 \text{ feet}} \\
	&= \frac{12.3 \text{ feet}}{1} \cdot \frac{1 \text{ yard}}{3 \text{ feet}} \\
	&= \frac{12.3}{1} \cdot \frac{1 \text{ yard}}{3} \\
	&= \frac{12.3 \cdot 1}{1 \cdot 3} \text{ yards} \\
	&= \frac{12.3}{3} \text{ yards} \\
	&= 4.1 \text{ yards}
\end{aligned}
\]

\textbf{Conclusion:} $12.3$ feet is equivalent to $4.1$ yards.

\textbf{Method 3:} We will simply use multiplication/division to do this problem. It's given $1$ yard = $3$ feet. When we convert $12.3$ feet to yards, the number will become smaller. That's why we should use division:
\[
\begin{aligned}[t]
	&\phantom{{}=} 12.3 \text{ feet} \\
	&= 12.3 \div 3 \text{ yards} \\
	&= 4.1 \text{ yards}
\end{aligned}
\]

\textbf{Conclusion:} $12.3$ feet is equivalent to $4.1$ yards.

\end{solution}

When more than one conversions are involved in the same problem, it's easier to use the second or the third method in the example above. Look at the next example.

\begin{myexample}
Convert $4.7$ yards to inches.
\end{myexample}
\begin{solution}

\textbf{Method 1:} We will use fraction multiplication to solve this problem. We will use the rates $\frac{3 \text{ feet}}{1 \text{ yard}}$ and $\frac{12 \text{ inches}}{1 \text{ foot}}$, so the units feet and inches will cancel:

\[
\begin{aligned}[t]
	&\phantom{{}=}4.7 \text{ yards} \cdot \frac{3 \text{ feet}}{1 \text{ yard}} \cdot \frac{12 \text{ inches}}{1 \text{ foot}}  \\
	&\phantom{{}=} \frac{4.7 \text{ yards}}{1} \cdot \frac{3 \text{ feet}}{1 \text{ yard}} \cdot \frac{12 \text{ inches}}{1 \text{ foot}}  \\
	&= \frac{4.7}{1} \cdot \frac{3}{1} \cdot \frac{12 \text{ inches}}{1} \\
	&= 4.7 \cdot 3 \cdot 12 \text{ inches} \\
	&= 169.2 \text{ inches}
\end{aligned}
\]

\textbf{Conclusion:} $4.7$ yards is equivalent to $169.2$ inches.

\textbf{Method 2:} We will simply use multiplication/division to do this problem. When we convert yards to inches, the number will become bigger, so we do multiplication twice:
\[
\begin{aligned}[t]
	&\phantom{{}=} 4.7 \text{ yards} \\
	&= 4.7 \cdot 3 \text{ feet} \\
	&= 14.1 \text{ feet} \\
	&= 14.1 \cdot 12 \text{ inches} \\
	&= 169.2 \text{ inches}
\end{aligned}
\]

\textbf{Conclusion:} $4.7$ yards is equivalent to $169.2$ inches.

\end{solution}

\subsection{Metric System Conversions}
Metric system conversions are easier, because the factor is always by the scale of $10$. For example:
\begin{itemize}
\item $1$ meter = $10$ decimeters
\item $1$ decimeter = $10$ centimeters
\item $1$ centimeter = $10$ millimeters
\item $1$ meter = $100$ centimeters
\item $1$ meter = $1000$ millimeters
\item $1$ kilometer = $1000$ meters
\item $1$ kilogram = $1000$ grams
\item $1$ gram = $1000$ milligrams
\end{itemize}

Understand the following:
\begin{itemize}
\item kilo- means $1000$
\item deci- means $10$
\item centi- means $100$
\item milli- means $1000$
\end{itemize}

Because of this, the easiest method to convert metric system is to use multiplication/division. Let's look at a few examples.

\begin{myexample}
Convert $1,234$ meters to kilometers.
\end{myexample}
\begin{solution}
Since $1$ kilometer = $1,000$ meters, to change $1,234$ meters to kilometers, the number will become smaller. We will do division:
\[
\begin{aligned}[t]
	&\phantom{{}=} 1234 \text{ meters} \\
	&= 1234 \div 1000 \text{ kilometers} \\
	&= 1.234 \text{ kilometers}
\end{aligned}
\]

\textbf{Conclusion:} $1,234$ meters is equivalent to $1.234$ kilometers.
\end{solution}

\begin{myexample}
Convert $0.32$ meter to centimeters.
\end{myexample}
\begin{solution}
Since $1$ meter = $100$ centimeters, to change $0.32$ meters to centimeters, the number will become bigger. We will do multiplication:
\[
\begin{aligned}[t]
	&\phantom{{}=} 0.32 \text{ meters} \\
	&= 0.32 \cdot 100 \text{ centimeters} \\
	&= 32 \text{ centimeters}
\end{aligned}
\]

\textbf{Conclusion:} $0.32$ meter is equivalent to $32$ centimeters.
\end{solution}

\subsection{Time Units Conversions}
Everyone should know the following conversions:

\begin{itemize}
\item $1$ week = $7$ days
\item $1$ day = $24$ hours
\item $1$ hour = $60$ minutes
\item $1$ minute = $60$ seconds
\end{itemize}

When doing time units conversions, we can use either fraction multiplication method or multiplication/division.

\begin{myexample}
Convert $3,024,000$ seconds into weeks.
\end{myexample}
\begin{solution}

\textbf{Method 1:} We will use fraction multiplication to solve this problem:

\[
\begin{aligned}[t]
	&\phantom{{}=}3024000 \text{ seconds} \cdot \frac{1 \text{ minute}}{60 \text{ seconds}} \cdot \frac{1 \text{ hour}}{60 \text{ minutes}} \cdot \frac{1 \text{ day}}{24 \text{ hours}} \cdot \frac{1 \text{ week}}{7 \text{ days}} \\
	&=\frac{3024000 \text{ seconds}}{1} \cdot \frac{1 \text{ minute}}{60 \text{ seconds}} \cdot \frac{1 \text{ hour}}{60 \text{ minutes}} \cdot \frac{1 \text{ day}}{24 \text{ hours}} \cdot \frac{1 \text{ week}}{7 \text{ days}} \\
	&=\frac{3024000}{1} \cdot \frac{1}{60} \cdot \frac{1}{60} \cdot \frac{1}{24} \cdot \frac{1 \text{ week}}{7} \\
	&= \frac{3024000 \cdot 1 \cdot 1 \cdot 1 \cdot 1}{1 \cdot 60 \cdot 60 \cdot 24 \cdot 7} \text{ weeks} \\
	&= \frac{3024000}{604800} \text{ weeks} \\
	&= 5 \text{ weeks}
\end{aligned}
\]

\textbf{Conclusion:} $3,024,000$ seconds is equivalent to $5$ weeks.

\textbf{Solution 2:} We will simply use multiplication/division to do this problem. When we convert $3,024,000$ seconds to weeks, the number will become smaller, so we will do divisions:
\[
\begin{aligned}[t]
	&\phantom{{}=} 3024000 \text{ seconds} \\
	&= 3024000 \div 60 \text{ minutes} \\
	&= 50400 \text{ minutes} \\
	&= 50400 \div 60 \text{ hours} \\
	&= 840 \text{ hours} \\
	&= 840 \div 24 \text{ days} \\
	&= 35 \text{ days} \\
	&= 35 \div 7 \text{ weeks} \\
	&= 5 \text{ weeks}
\end{aligned}
\]

\textbf{Conclusion:} $3,024,000$ seconds is equivalent to $5$ weeks.

\end{solution}

