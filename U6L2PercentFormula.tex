
\section{Percent Formula}

In this lesson, we will learn $3$ types of percent problems, and use $3$ methods to solve each type of problems. You can choose your favorite method. It would be great if you can use all methods, so you can have a better understanding of percent problems.

We will handle the following $3$ types of percent problems:
\begin{itemize}
\item \textbf{Type I:} What is $40\%$ of $20$?
\item \textbf{Type II:} $8$ is what percent of $20$?
\item \textbf{Type III:} $8$ is $40\%$ of what?
\end{itemize}

The Percent Formula is one popular method to solve the above problems.

\subsection{Percent Formula}
Many textbooks give the Percent Formula as:
\begin{center}
Part = Percent $\cdot$ Whole
\end{center}
or something similar. In this lesson, instead of memorizing a formula, I expect you to write out the following example on scratch paper.

Recall that the English word "of" can be translated into the multiplication symbol, in situations like "$3$ is $\frac{1}{2}$ of $6$." You can verify this by doing:
\[ \frac{1}{2} \cdot 6 = \frac{1}{2} \cdot \frac{6}{1} = \frac{6}{2} = 3 \]

It's common sense that $\frac{1}{2}=50\%$, so we have "$3$ is $50\%$ of $6$." Here is the percent formula:
\[ 3 = 50\% \cdot 6 \]
Once we change $50\%$ to $0.5$, we can verify $0.5\cdot6=3$ with calculator.

When you need to use the Percent Formula, instead of writing down a formula, write down the example above. In later lessons, when I refer to the "Percent Formula", I mean: 
\[ 3 = 50\% \cdot 6 \]

There are three types of percent problems.

\subsection{Type I Percent Problem}
\begin{myexample}
What is $40\%$ of $20$?
\end{myexample}
\begin{solution}

\textbf{Method 1: Use Percent Formula} Assume $x$ is $40\%$ of $20$. We will write down the "Percent Formula" and the problem right next to each other:
\[
\begin{aligned}[t]
	&3 &= &&50\% &&\cdot &&6 \\
	&x &= &&40\% &&\cdot &&20
\end{aligned}
\]

Next, we can solve for $x$ in the equation. In this type of problems, $x$ happens to be alone on one side of the equal sign, so all we need to do is to do the calculation on the other side of the equal sign. Remember that $40\%=0.4$. We have:
\[
\begin{aligned}[t]
	x &= 40\% \cdot 20 \\
	x &= 0.4 \cdot 20 \\
	x &= 8
\end{aligned}
\]
\textbf{Conclusion:} $8$ is $40\%$ of $20$.

\textbf{Method 2: Use Proportion}  Assume $x$ is $40\%$ of $20$. We can rephrase this sentence as: $x$ out of $20$ is like $40$ out of $100$. Here is the key: The number following the word "of" corresponds to $100$. Now we can set up and solve a proportion:
\[
\begin{aligned}[t]
	\frac{x}{20} &= \frac{40}{100} \\
	100x &= 20 \cdot 40 \\
	100x &= 800 \\
	\frac{100x}{100} &= \frac{800}{100} \\
	x &= 8
\end{aligned}
\]
\textbf{Conclusion:} $8$ is $40\%$ of $20$.

\textbf{Method 3: Use Multiplication/Division} No variable ($x$) is involved in this method. The key is to write down a simple example on scratch paper, and then put numbers in their corresponding places.

To find "$50\%$ of $6$", we do:
\[ 50\% \cdot 6=0.5\cdot6=3 \]
Similarly, to find "$40\%$ of $20$", we do:
\[ 40\% \cdot 20=0.4\cdot20=8 \]
\textbf{Conclusion:} $8$ is $40\%$ of $20$.

\end{solution}

\subsection{Type II Percent Problem}
\begin{myexample}
$8$ is what percent of $20$?
\end{myexample}
\begin{solution}
\textbf{Method 1: Use Percent Formula} Assume $8$ is $x$ (as a percent) of $20$. We will write down the "Percent Formula" and the problem right next to each other:
\[
\begin{aligned}[t]
	&3 &= &&50\% &&\cdot &&6 \\
	&8 &= &&x \text{ (as a percent)} &&\cdot &&20
\end{aligned}
\]

Next, we can solve for $x$ in the equation:
\[
\begin{aligned}[t]
	8 &= x \cdot 20 \\
	8 &= 20x \\
	\frac{8}{20} &= \frac{20x}{20} \\
	0.4 &= x \\
	40\% &= x
\end{aligned}
\]
\textbf{Conclusion:} $8$ is $40\%$ of $20$.

\textbf{Method 2: Use Proportion}  Assume $8$ is $x\%$ of $20$. We can rephrase this sentence as: $8$ out of $20$ is like $x$ out of $100$. Here is the key: The number following the word "of" corresponds to $100$. Now we can set up and solve a proportion:
\[
\begin{aligned}[t]
	\frac{8}{20} &= \frac{x}{100} \\
	20x &= 8 \cdot 100 \\
	20x &= 800 \\
	\frac{20x}{20} &= \frac{800}{20} \\
	x &= 40
\end{aligned}
\]
\textbf{Conclusion:} $8$ is $40\%$ of $20$.

\textbf{Method 3: Use Multiplication/Division} No variable ($x$) is involved in this method. The key is to write down a simple example on scratch paper, and then put numbers in their corresponding places.

To find "$3$ is what percent of $6$", we do:
\[ 3\div6=0.5=50\% \]
Similarly, to find "$8$ is what percent of $20$", we do:
\[ 8\div20=0.4=40\% \]
\textbf{Conclusion:} $8$ is $40\%$ of $20$.

\end{solution}

\subsection{Type III Percent Problem}
\begin{myexample}
$8$ is $40\%$ of what?
\end{myexample}
\begin{solution}

\textbf{Method 1: Use Percent Formula} Assume $8$ is $40\%$ of $x$. We will write down the "Percent Formula" and the problem right next to each other:
\[
\begin{aligned}[t]
	&3 &= &&50\% &&\cdot &&6 \\
	&8 &= &&40\% &&\cdot &&x
\end{aligned}
\]

Next, we can solve for $x$ in the equation:
\[
\begin{aligned}[t]
	8 &= 40\% \cdot x \\
	8 &= 0.4x \\
	\frac{8}{0.4} &= \frac{0.4x}{0.4} \\
	20 &= x
\end{aligned}
\]
\textbf{Conclusion:} $8$ is $40\%$ of $20$.

\textbf{Method 2: Use Proportion}  Assume $8$ is $40\%$ of $x$. We can rephrase this sentence as: $8$ out of $x$ is like $40$ out of $100$. Here is the key: The number following the word "of" corresponds to $100$. Now we can set up and solve a proportion:
\[
\begin{aligned}[t]
	\frac{8}{x} &= \frac{40}{100} \\
	40x &= 8 \cdot 100 \\
	40x &= 800 \\
	\frac{40x}{40} &= \frac{800}{40} \\
	x &= 20
\end{aligned}
\]
\textbf{Conclusion:} $8$ is $40\%$ of $20$.

\textbf{Method 3: Use Multiplication/Division} No variable ($x$) is involved in this method. The key is to write down a simple example on scratch paper, and then put numbers in their corresponding places.

To find "$3$ is $50\%$ of what", we do:
\[ 3\div0.5=6 \]
Similarly, to find "$8$ is $40\%$ of what", we do:
\[ 8\div0.4=20 \]
\textbf{Conclusion:} $8$ is $40\%$ of $20$.

\end{solution}

\subsection{Rounding}
Sometimes we need to round numbers, like in the next example.
\begin{myexample}
$45$ is what percent of $981$? Round your answer to two decimal places.
\end{myexample}
\begin{solution}

This is a Type II percent problem. We will use the Percent Formula to solve this problem. Assume $45$ is $x$ (as a percent) of $981$.

We will write down the "Percent Formula" and the problem right next to each other:
\[
\begin{aligned}[t]
	&3 &= &&50\% &&\cdot &&6 \\
	&45 &= &&x \text{ (as a percent)} &&\cdot &&981
\end{aligned}
\]

Next, we can solve for $x$ in the equation:
\[
\begin{aligned}[t]
	45 &= x \cdot 981 \\
	45 &= 981x \\
	\frac{45}{981} &= \frac{981x}{981} \\
	0.0459 &\approx x \\
	4.59\% &\approx x
\end{aligned}
\]
\textbf{Conclusion:} $45$ is approximately $4.59\%$ of $981$.

\end{solution}

\subsection{More than $100\%$}
The rules are the same when we deal with percents bigger than $100\%$. Let's look at a few examples. To save space, we will solve these problems only with the Percent Formula.

\begin{myexample}
What is $140\%$ of $20$?
\end{myexample}
\begin{solution}

We know the solution must be bigger than $20$, because the percent is bigger than $100\%$.

Assume $x$ is $140\%$ of $20$. We will write down the "Percent Formula" and the problem right next to each other:
\[
\begin{aligned}[t]
	&3 &= &&50\% &&\cdot &&6 \\
	&x &= &&140\% &&\cdot &&20
\end{aligned}
\]

Next, we can solve for $x$ in the equation. In this type of problems, $x$ happens to be alone on one side of the equal sign, so all we need to do is to do the calculation on the other side of the equal sign. We have:
\[
\begin{aligned}[t]
	x &= 140\% \cdot 20 \\
	x &= 1.4 \cdot 20 \\
	x &= 28
\end{aligned}
\]
\textbf{Conclusion:} $28$ is $140\%$ of $20$.

\end{solution}

\begin{myexample}
$28$ is what percent of $20$?
\end{myexample}
\begin{solution}
We know the solution must be bigger than $100\%$, because $28$ is bigger than $20$.

Assume $28$ is $x$ (as a percent) of $20$. We will write down the "Percent Formula" and the problem right next to each other:
\[
\begin{aligned}[t]
	&3 &= &&50\% &&\cdot &&6 \\
	&28 &= &&x \text{ (as a percent)} &&\cdot &&20
\end{aligned}
\]

Next, we can solve for $x$ in the equation:
\[
\begin{aligned}[t]
	28 &= x \cdot 20 \\
	28 &= 20x \\
	\frac{28}{20} &= \frac{20x}{20} \\
	1.4 &= x \\
	140\% &= x
\end{aligned}
\]
\textbf{Conclusion:} $28$ is $140\%$ of $20$.
\end{solution}

\begin{myexample}
$28$ is $140\%$ of what?
\end{myexample}
\begin{solution}
We know the solution must be smaller than $28$, because the percent is bigger than $100\%$.

Assume $28$ is $140\%$ of $x$. We will write down the "Percent Formula" and the problem right next to each other:
\[
\begin{aligned}[t]
	&3 &= &&50\% &&\cdot &&6 \\
	&28 &= &&140\% &&\cdot &&x
\end{aligned}
\]

Next, we can solve for $x$ in the equation:
\[
\begin{aligned}[t]
	28 &= 140\% \cdot x \\
	28 &= 1.4x \\
	\frac{28}{1.4} &= \frac{1.4x}{1.4} \\
	20 &= x
\end{aligned}
\]
\textbf{Conclusion:} $28$ is $140\%$ of $20$.

\end{solution}
