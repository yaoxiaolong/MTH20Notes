
\section{Add/Subtract Decimals}

In this section, we will learn how to add/subtract decimals. 

Although we could use calculator to do decimal calculations, it's important to understand the theory and do some calculations without using calculator. Please refrain from using calculator in this section (actually in this whole course).

In most cases, we add/subtract decimals in the "normal" way.

\begin{tabular}[t]{c@{\hspace{4cm}}c@{\hspace{2cm}}c}
&
$ \begin{aligned}[t] 
	&\phantom{+}12.34 \\
	&\underline{+23.45} \\
	&\phantom{+}35.79
  \end{aligned} $ 
&
$ \begin{aligned}[t] 
	&\phantom{-}98.76 \\
	&\underline{-23.45} \\
	&\phantom{-}75.31
  \end{aligned} $ 
\end{tabular}

However, it takes some thinking to do $0.2+0.03$. A common mistake is to do $0.2+0.03=0.5$. Let's think about money.

The number $0.2$ means $20$ cents; the number $0.03$ means $3$ cents; so $0.2+0.03$ should be $23$ cents, or $0.2+0.03=0.23$. Here is how we should line up $0.2$ and $0.03$:

\[
\begin{aligned}[t] 
	&\phantom{+}0.2 \\
	&\underline{+0.03} \\
	&\phantom{+}0.23
\end{aligned}
\]

Here is the rule: When we add/subtract decimals, we must line up the decimal point.

When we subtract decimals, sometimes we need to fill in some zeros at the end. For example, when we calculate $0.2-0.03$, we do:

\[
\begin{aligned}[t] 
	&\phantom{-}0.20 \\
	&\underline{-0.03} \\
	&\phantom{-}0.17
\end{aligned}
\]

This result makes sense, as $20$ cents minus $3$ cents is $17$ cents.

For integers like $2$, the decimal point is at the end, even though we don't write it. When we do $2+1.4$, we first change $2$ to $2.0$, and then line up:

\[
\begin{aligned}[t] 
	&\phantom{+}2.0 \\
	&\underline{+1.4} \\
	&\phantom{+}3.4
\end{aligned}
\]

When we do $2-0.04$, we change $2$ to $2.00$, and then line up:

\[
\begin{aligned}[t] 
	&\phantom{-}2.00 \\
	&\underline{-0.04} \\
	&\phantom{-}1.96
\end{aligned}
\]

It's more complicated when negative decimals are involved. We will do two problems with the "money model" (instead of the "number line" model):

\begin{myexample}
Calculate $1.907-2.3$
\end{myexample}
\begin{solution}
\begin{enumerate}
\item Change $1.907-2.3$ to $1.907+(-2.3)$.
\item The first number is 1.907. For the money model, in the first game, I won \$$1.907$.
\item The second number is $-2.3$. For the money model, in the second game, I lost \$$2.3$.
\item Since I lost more money than I won, altogether, I lost some money. This implies the answer must be negative.
\item Since I won some and then lost some, we need to find the difference of $1.907$ and $2.3$. We do:
\[
\begin{aligned}[t] 
	&\phantom{-}2.300 \\
	&\underline{-1.907} \\
	&\phantom{-}0.393
\end{aligned}
\]
\item[6.] Finally, the solution is: $1.907-2.3=-0.393$
\end{enumerate}
\end{solution}

\begin{myexample}
Calculate $-1.907-2.3$
\end{myexample}
\begin{solution}
\begin{enumerate}
\item Change $-1.907-2.3$ to $(-1.907)+(-2.3)$.
\item The first number is $-1.907$. For the money model, in the first game, I lost \$$1.907$.
\item The second number is $-2.3$. For the money model, in the second game, I lost \$$2.3$.
\item Since I lost in both games, altogether, I lost money. This implies the answer must be negative.
\item Since I lost in both games, we need to find the sum of $1.907$ and $2.3$. We do:
\[
\begin{aligned}[t] 
	&\phantom{+}2.300 \\
	&\underline{+1.907} \\
	&\phantom{+}4.207
\end{aligned}
\]
\item[6.] Finally, the solution is: $-1.907-2.3=-4.207$
\end{enumerate}
\end{solution}

