
\section{Divisibility Test}

Number sense is critical in the study of mathematics. The first step of building number sense is to memorize the multiplication table. In this lesson, we will learn how to tell whether a number goes into another evenly. For example, without doing division, we know $2$ goes into $126$ evenly (meaning $\frac{126}{2}$ doesn't have remainder). This lesson is also important in building number sense.

\subsection{Divisibility of $2$}
Let's look at the first few multiples of $2$:
\[ 2, 4, 6, 8, 10, 12, 14, 16, 18, 20, 22, 24, ... \]

We can see they are all even numbers, so $2$ goes into all even numbers evenly. In other words, $2$ goes into a number evenly if its last digit is $0,2,4,6 \text{ or } 8$.

\subsection{Divisibility of $10$}
Let's look at some multiples of $10$:
\[ 10, 20, 30, 40, 50, ..., 100, 110, ..., 1230, 1240, ... \]

It's easy to see that $10$ goes evenly into all numbers whose last digit is $0$.

\subsection{Divisibility of $5$}
Let's look at some multiples of $5$:
\[ 5, 10, 15, 20, 25, 30, ..., 100, 105, 110, ..., 1230, 1235, ... \]

It's easy to see that $5$ goes evenly into all numbers whose last digit is $0$ or $5$.

\subsection{Divisibility of $3$ and $9$}
Let's look at some multiples of $3$:
\[ 3,6,9,12,15,...,99,102,105,...,300,303,306,309,312,...3333,3336,3339,3342,... \]

Let's add up all digits of some of these numbers:
\[
\begin{aligned}[t]
   &15:\: 1+5=6 \\
   &102:\: 1+0+2=3 \\
   &309:\:3+0+9=12 \\
   &3342:\: 3+3+4+2=15
\end{aligned}
\]

Notice that $3$ goes evenly into all these sums: $6,3,12,15$.

To judge whether $3$ goes into a number evenly, we add up all digits of this number. If $3$ goes into the sum evenly, $3$ goes into the number evenly.

\begin{myexample}
Does $3$ go into $195$ and $1984$ evenly?
\end{myexample}
\begin{solution}
We add up all digits of $195$, and we have $1+9+5=15$. Since $3$ goes into $15$ evenly, $3$ must go into $195$ evenly. We can use a calculator to verify this: $\frac{195}{3}=65$.

We add up all digits of $1984$, and we have $1+9+8+4=22$. Since $3$ does not go into $22$ evenly, $3$ does not go into $1994$ evenly. We can use a calculator to verify this: $\frac{1984}{3}=661.333...$.
\end{solution}

This rule also works for $9$, but doesn't work for any other numbers.

\begin{myexample}
Does $9$ go into $396$ and $987$ evenly?
\end{myexample}
\begin{solution}
We add up all digits of $396$, and we have $3+9+6=18$. Since $9$ goes into $18$ evenly, $9$ must go into $396$ evenly. We can use a calculator to verify this: $\frac{396}{9}=44$.

We add up all digits of $987$, and we have $9+8+7=24$. Since $9$ does not go into $24$ evenly, $9$ does not go into $987$ evenly. We can use a calculator to verify this: $\frac{987}{9}=109.666...$.
\end{solution}

Again, don't try to use this rule for $4$, $5$, or any number other than $3$ and $9$. Next, let's review all divisibility rules we learned.

\begin{myexample}
Decide whether $2$, $3$, $5$, $9$ and $10$ go into $12,345$ evenly.
\end{myexample}
\begin{solution}
\begin{itemize}
\item Since $12,345$ is odd, $2$ does not go into $12,345$ evenly.
\item Add up all digits of $12,345$, we have $1+2+3+4+5=15$. Since $3$ goes into $15$ evenly, $3$ must go into $12,345$ evenly.
\item Since the last digit of $12,345$ is $5$, $5$ must go into $12,345$ evenly.
\item Add up all digits of $12,345$, we have $1+2+3+4+5=15$. Since $9$ does not go into $15$ evenly, $9$ does not go into $12,345$ evenly.
\item Since the last digit of $12,345$ is $5$, $10$ does not go into $12,345$ evenly.
\end{itemize}
\end{solution}

\begin{myexample}
Decide whether $2$, $3$, $5$, $9$ and $10$ go into $65,430$.
\end{myexample}
\begin{solution}
\begin{itemize}
\item Since $65,430$ is even, $2$ goes into $65,430$ evenly.
\item Add up all digits of $65,430$, we have $6+5+4+3+0=18$. Since $3$ goes into $18$ evenly, $3$ must go into $65,430$ evenly.
\item Since the last digit of $65,430$ is $0$, $5$ must go into $65,430$ evenly.
\item Add up all digits of $65,430$, we have $6+5+4+3+0=18$. Since $9$ goes into $18$ evenly, $9$ goes into $65,430$ evenly.
\item Since the last digit of $65,430$ is $0$, $10$ goes into $65,430$ evenly.
\end{itemize}
\end{solution}

\subsection{Summary}
Let's review important concepts in this lesson.
\begin{itemize}
\item $2$ goes into all even numbers (ending with $0,2,4,6 \text{ or} 8$).
\item If $3$ goes evenly into the sum of all digits of a number, $3$ goes into the number evenly.
\item $5$ goes evenly into all numbers whose last digit is $5$ or $10$.
\item If $9$ goes evenly into the sum of all digits of a number, $9$ goes into the number evenly.
\item $10$ goes evenly into all numbers whose last digit is $10$.
\end{itemize}

Granted, if the question is: "Can $3$ go into $123$ evenly?", you could use the calculator to do the division $123\div3$ and see whether the quotient is an integer or decimal. However, the content of this lesson is very important in building your number sense, which is very important in later mathematics study.

