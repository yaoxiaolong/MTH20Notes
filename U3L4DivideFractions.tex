
\section{Divide Fractions}
\thispagestyle{fancy}

In this lesson, we will learn how to divide fractions.

\subsection{Understand Dividing Fractions}

How many quarters are there in \$$3.00$? This is a division problem:
\[ 3\div\frac{1}{4} \]
To do this problem, we divide all \$$3.00$ into quarters, and we will have $3\cdot4=12$ quarters. So the solution is $3\div\frac{1}{4}=12$.

Next, let's make this example more interesting:
\[ 3\div\frac{3}{4} \]
We need to find how many "three quarters" are in \$$3.00$? We will do this in two steps:
\begin{enumerate}
\item We divide $3$ into quarters, and we have $3\cdot4=12$ quarters.
\item We divide $12$ quarters into groups of $3$, and we have $12\div3=4$ groups.
\end{enumerate}
The solution is:
\[ 3\div\frac{3}{4}=3\cdot4\div3=12\div3=4 \]
Recall that the fraction line is the same as the division symbol. We can re-write the solution as:
\[ 3\div\frac{3}{4}=3\cdot\frac{4}{3}=\frac{3}{1}\cdot\frac{4}{3}=\frac{12}{3}=4 \]
The second way is how most middle school teachers teach dividing fractions: We change division to multiplication, and at the same time "flip" the second fraction. Unfortunately, the procedure doesn't make sense. I hope the money model helps you understand why the procedure makes sense.

I have to say it's rather difficult to understand fraction division. If you are not interested in understanding why, simply remember the rule.

\begin{myexample}
\[ 
\begin{aligned}[t]
	&\phantom{{}=}\frac{2}{3} \div \frac{2}{9} \\
	&= \frac{2}{3} \cdot \frac{9}{2} \\
	&= \frac{2\div2}{3\div3} \cdot \frac{9\div3}{2\div2} \\
	&= \frac{1}{1} \cdot \frac{3}{1} \\
	&= \frac{3}{1} \\
	&= 3
\end{aligned}
\]
Never let fractions like $\frac{3}{1}$ be the final answer! Change it to an integer.
\end{myexample}

\subsection{Dividing Fractions involving Integers}
If an integer is involved in fraction division, we first change the integer to a fraction by adding $1$ to the denominator. This is similar to the rule for fraction multiplication.

\begin{myexample}
\[ 
\begin{aligned}[t]
	&\phantom{{}=} 10 \div \frac{5}{9} \\
	&= \frac{10}{1} \div \frac{5}{9} \\
	&= \frac{10}{1} \cdot \frac{9}{5} \\
	&= \frac{10\div5}{1} \cdot \frac{9}{5\div5} \\
	&= \frac{2}{1} \cdot \frac{9}{1} \\
	&= \frac{18}{1} \\
	&= 18
\end{aligned}
\]
\end{myexample}

\begin{myexample}
\[ 
\begin{aligned}[t]
	&\phantom{{}=} \frac{5}{9} \div 10 \\
	&= \frac{5}{9} \div \frac{10}{1} \\
	&= \frac{5}{9} \cdot \frac{1}{10} \\
	&= \frac{5\div5}{9} \cdot \frac{1}{10\div5} \\
	&= \frac{1}{9} \cdot \frac{1}{2} \\
	&= \frac{1}{18}
\end{aligned}
\]
\end{myexample}

\subsection{Fraction Division Word Problems}
\begin{myexample}
A school won a grant, and will spend $\frac{2}{3}$ of it to purchase textbooks. Six committees will evenly share the textbook fund to purchase textbooks of $6$ subjects. What fraction of the total grant will be used by each committee to purchase textbooks?
\end{myexample}
\begin{solution}
In this problem, $\frac{2}{3}$ will be evenly divided into $6$ pieces, a division problem:
\[ 
\begin{aligned}[t]
	&\phantom{{}=} \frac{2}{3} \div 6 \\
	&= \frac{2}{3} \div \frac{6}{1} \\
	&= \frac{2}{3} \cdot \frac{1}{6} \\
	&= \frac{2\div2}{3} \cdot \frac{1}{6\div2} \\
	&= \frac{1}{3} \cdot \frac{1}{3} \\
	&= \frac{1}{9}
\end{aligned}
\]
\textbf{Conclusion:} Each committee will use $\frac{1}{9}$ of the total grant to purchase textbooks.
\end{solution}

\begin{myexample}
Jack will use his car to move a pile of rocks which weigh $\frac{7}{8}$ of a ton. Jack's car can load only $\frac{1}{16}$ of a ton. How many trips will it take to move the whole pile?
\end{myexample}
\begin{solution}
In this problem, we will repeatedly take away $\frac{1}{16}$ of a ton from $\frac{7}{8}$ of a ton, a division problem:
\[ 
\begin{aligned}[t]
	&\phantom{{}=} \frac{7}{8} \div \frac{1}{16} \\
	&= \frac{7}{8} \cdot \frac{16}{1} \\
	&= \frac{7}{8\div8} \cdot \frac{16\div8}{1} \\
	&= \frac{7}{1} \cdot \frac{2}{1} \\
	&= \frac{14}{1} \\
	&= 14
\end{aligned}
\]
\textbf{Conclusion:} It will take $14$ trips to move the whole pile of rocks.
\end{solution}


\subsection{Summary}
Let's review what we learned in this lesson:
\begin{itemize}
\item To divide two fractions, we change division to multiplication, and at the same time, flip the second fraction. For example: 
\[ \frac{1}{3} \div \frac{1}{2}= \frac{1}{3} \cdot \frac{2}{1} = \frac{2}{3} \]
\item If an integer is involved in fraction division, we first change the integer to a fraction, and then carry out the procedures above. For example:
\[ \frac{1}{3} \div 2 = \frac{1}{3} \div \frac{2}{1} = \frac{1}{3} \cdot \frac{1}{2} = \frac{1}{6} \]
\end{itemize}

