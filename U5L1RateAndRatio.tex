

\chapter{Rate and Proportion}
\thispagestyle{fancy}

It's critical to understand rate and proportion, as we use the concept every day. For example, if it takes $2\frac{1}{4}$ cups of flour to make three servings of food, how many cups of flour should be used to make eight servings? In super market, if $6$ oz of coffee costs $7.99$, while $9$ oz of coffee costs $9.99$, which choice is cheaper?

\section{Rate and Ratio}

\subsection{Rate}
In a rate, we always see the word "per", "each", "every" or simply the number $1$. We use division to calculate rate. It's important to include units in the calculation. Let's look at a few examples.

\begin{myexample}
A car drove $150$ miles in $6$ hours. What's the car's speed in miles per hour?
\end{myexample}
\begin{solution}
We use division to find the rate of change (speed in this case):
\[ \frac{150 \text{ miles}}{6 \text{ hours}}=25 \text{ miles/hour} \]
The car's speed is $25$ miles per hour.
\end{solution}

\begin{myexample}
A car drove $150$ miles in $6$ hours. How long does it take the car to travel each mile?
\end{myexample}
\begin{solution}
We use division to find the rate of change:
\[ \frac{6 \text{ hours}}{150 \text{ miles}}=0.04 \text{ hour/mile} \]
It takes the car $0.04$ hour to drive each mile. Later we will learn that $0.04$ hour is $144$ seconds.
\end{solution}

We can see $25 \text{ miles/hour}$ and $0.04 \text{ hour/mile}$ are equivalent.

Compare those two examples above, you can see why it's important to include units in the calculation of rate. In real life, speed is regularly used, but sometimes we need to use the other rate: $0.04$ hours per mile.

\subsection{Problem Solving with Rate}
When we use rate to solve problems, the key is to include units in the calculation. Let's look an example:

\begin{myexample}
A car can drive $25$ miles per hour.
\begin{enumerate}
\item How long does it take the car to drive $300$ miles?
\item How far can the car go in $30$ hours?
\end{enumerate}
\end{myexample}
\begin{solution}
We could use multiplication and division to solve this problem. However, we will use fraction multiplication, which is an important skill later when we study unit conversions.

The rate is given as $25$ miles per hour. This can be written in two ways: $\frac{25 \text{ miles}}{1 \text{ hour}}$ or $\frac{1 \text{ hour}}{25 \text{ miles}}$.

\begin{enumerate}
\item How long does it take the car to drive $300$ miles?

We need to use the rate $\frac{1 \text{ hour}}{25 \text{ miles}}$, so the unit "miles" will cancel:
\[
\begin{aligned}[t]
	&\phantom{{}=}300 \text{ miles} \cdot \frac{1 \text{ hour}}{25 \text{ miles}} \\
	&= \frac{300 \text{ miles}}{1} \cdot \frac{1 \text{ hour}}{25 \text{ miles}} \\
	&= \frac{300}{1} \cdot \frac{1 \text{ hour}}{25} \\
	&= \frac{300 \cdot 1}{1 \cdot 25} \text{ hours} \\
	&= \frac{300}{25} \text{ hours} \\
	&= 12 \text{ hours}
\end{aligned}
\]

It takes the car $12$ hours to drive $300$ miles.
\item How far can the car go in $30$ hours?

We need to use the rate $\frac{25 \text{ miles}}{1 \text{ hour}}$, so the unit "hours" will cancel:
\[
\begin{aligned}[t]
	&\phantom{{}=}30 \text{ hours} \cdot \frac{25 \text{ miles}}{1 \text{ hour}} \\
	&= \frac{30 \text{ hours}}{1} \cdot \frac{25 \text{ miles}}{1 \text{ hour}} \\
	&= \frac{30}{1} \cdot \frac{25 \text{ miles}}{1} \\
	&= 30 \cdot 25 \text{ miles} \\
	&= 750 \text{ miles}
\end{aligned}
\]

The car can drive $750$ miles in $30$ hours.
\end{enumerate}
\end{solution}

From the first two examples in this lesson, we learned that $25 \text{ miles/hour}$ and $0.04 \text{ hour/mile}$ are equivalent. In the next example, we will solve the same problem with the rate $0.04 \text{ hour/mile}$.

\begin{myexample}
It takes a car $0.04$ hour to drive a mile.
\begin{enumerate}
\item How long does it take the car to drive $300$ miles?
\item How far can the car go in $30$ hours?
\end{enumerate}
\end{myexample}
\begin{solution}

The rate is given as $0.04$ hour per mile. This can be written in two ways: $\frac{0.04 \text{ hour}}{1 \text{ mile}}$ or $\frac{1 \text{ mile}}{0.04 \text{ hour}}$.

\begin{enumerate}
\item How long does it take the car to drive $300$ miles?

We need to use the rate $\frac{0.04 \text{ hour}}{1 \text{ mile}}$, so the unit "mile" will cancel:
\[
\begin{aligned}[t]
	&\phantom{{}=}300 \text{ miles} \cdot \frac{0.04 \text{ hour}}{1 \text{ mile}} \\
	&= \frac{300 \text{ miles}}{1} \cdot \frac{0.04 \text{ hour}}{1 \text{ mile}} \\
	&= \frac{300}{1} \cdot \frac{0.04 \text{ hour}}{1} \\
	&= 300 \cdot 0.04 \text{ hours} \\
	&= 12 \text{ hours}
\end{aligned}
\]

It takes the car $12$ hours to drive $300$ miles.
\item How far can the car go in $30$ hours?

We need to use the rate $\frac{1 \text{ mile}}{0.04 \text{ hour}}$, so the unit "hour" will cancel:
\[
\begin{aligned}[t]
	&\phantom{{}=}30 \text{ hours} \cdot \frac{1 \text{ mile}}{0.04 \text{ hour}} \\
	&= \frac{30 \text{ hours}}{1} \cdot \frac{1 \text{ mile}}{0.04 \text{ hour}} \\
	&= \frac{30}{1} \cdot \frac{1 \text{ miles}}{0.04} \\
	&= \frac{30 \cdot 1}{1 \cdot 0.04} \text{ miles} \\
	&= \frac{30}{0.04} \text{ miles} \\
	&= 750 \text{ hours}
\end{aligned}
\]

The car can drive $750$ miles in $30$ hours.
\end{enumerate}
\end{solution}

\subsection{Ratio}
Ratio is very similar to rate, except it doesn't have unit. Let's look at a few examples.

\begin{itemize}
\item If a class has $30$ male students and $10$ female students, the ratio of male to female students is $\frac{30 \text{ people}}{10 \text{ people}}=\frac{30}{10}=\frac{3}{1}$. We can also say the ratio of male to female students is $3:1$, or "$3$ to $1$".
\item If a class has $30$ male students and $10$ female students, the ratio of female to male students is $\frac{10 \text{ people}}{30 \text{ people}}=\frac{10}{30}=\frac{1}{3}$. We can also say the ratio of female to male students is $1:3$, or "$1$ to $3$".
\item If Tom makes \$$15.00$ per hour and Jerry makes \$$12.00$ per hour, the ratio of Tom and Jerry's income is $\frac{15 \text{ dollars/hour}}{12 \text{ dollars/hour}}=\frac{15}{12}=\frac{5}{4}$. We can also say the ratio of Tom and Jerry's income is $5:4$, or "$5$ to $4$".
\item If Tom makes \$$15.00$ per hour and Jerry makes \$$12.00$ per hour, the ratio of Jerry and Tom's income is $\frac{12 \text{ dollars/hour}}{15 \text{ dollars/hour}}=\frac{12}{15}=\frac{4}{5}$. We can also say the ratio of Jerry and Tom's income is $4:5$, or "$4$ to $5$".
\end{itemize}

Note that the units always cancel in a ratio. If the units don't cancel, like in $25$ miles/hour, then it's called a rate, not a ratio. That's the major difference between rate and ratio.

\subsection{Problem Solving with Ratio}
When we use ratio to solve word problems, it's important not only to include units, but also the concepts. Let's look at some examples.

\begin{myexample}
A restaurant's expense of labor cost to food cost is in the ratio of $8:3$.
\begin{enumerate}
\item In one month, if the restaurant spent \$$2,000.00$ in labor cost, how much did it spend on food cost?
\item In another month, if the restaurant spent \$$600.00$ in food cost, how much did it spend on labor cost?
\end{enumerate}
\end{myexample}
\begin{solution}

The ratio of labor cost to food cost is given as $8:3$. We can write it as either $\frac{8 \text{ dollars in labor cost}}{3 \text{ dollars in food cost}}$ or $\frac{3 \text{ dollars in food cost}}{8 \text{ dollars in labor cost}}$.

\begin{enumerate}
\item In one month, if the restaurant spent \$$2,000.00$ in labor cost, how much did it spend on food cost?

We need to use the ratio $\frac{3 \text{ dollars in food cost}}{8 \text{ dollars in labor cost}}$, so the unit "dollars in labor cost" will cancel:
\[
\begin{aligned}[t]
	&\phantom{{}=}2000 \text{ dollars in labor cost} \cdot \frac{3 \text{ dollars in food cost}}{8 \text{ dollars in labor cost}} \\
	&= \frac{2000 \text{ dollars in labor cost}}{1} \cdot \frac{3 \text{ dollars in food cost}}{8 \text{ dollars in labor cost}} \\
	&= \frac{2000}{1} \cdot \frac{3 \text{ dollars in food cost}}{8} \\
	&= \frac{2000\cdot3}{1\cdot8} \text{ dollars in food cost} \\
	&= \frac{6000}{8} \text{ dollars in food cost} \\
	&= 750 \text{ dollars in food cost}
\end{aligned}
\]

In one month, if the restaurant spent \$$2,000.00$ in labor cost, it spent \$$750$ on food cost.
\item In another month, if the restaurant spent \$$600.00$ in food cost, how much did it spend on labor cost?

We need to use the ratio $\frac{8 \text{ dollars in labor cost}}{3 \text{ dollars in food cost}}$, so the unit "dollars in food cost" will cancel:
\[
\begin{aligned}[t]
	&\phantom{{}=}600 \text{ dollars in food cost} \cdot \frac{8 \text{ dollars in labor cost}}{3 \text{ dollars in food cost}} \\
	&= \frac{600 \text{ dollars in food cost}}{1} \cdot \frac{8 \text{ dollars in labor cost}}{3 \text{ dollars in food cost}} \\
	&= \frac{600}{1} \cdot \frac{8 \text{ dollars in labor cost}}{3} \\
	&= \frac{600\cdot8}{1\cdot3} \text{ dollars in labor cost} \\
	&= \frac{4800}{3} \text{ dollars in labor cost} \\
	&= 1600 \text{ dollars in labor cost}
\end{aligned}
\]

In another month, if the restaurant spent \$$600.00$ in food cost, it spent \$$1,600$ on labor cost.
\end{enumerate}
\end{solution}

We can see why we need to write more than units in equations involving ratios---otherwise all units would be "dollars" in the above example, which would be confusing.

