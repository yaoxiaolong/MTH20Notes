
\chapter{Integer Operations}
\thispagestyle{fancy}
\section{Number Line and Absolute Value}

In this lesson, we will learn the concept of negative numbers, and their positions on the number line. We will also learn the concept of absolute value.

\subsection{Positive and Negative Numbers}

Negative numbers are regularly used in every day life. For example:

\begin{itemize}
\item The stock market lost $30.2$ points yesterday.
\item Today's temperature is $10$ degrees below zero.
\item Mr. Smith overdrew his bank account by \$$50$.
\item The wreck of Titanic is $12,420$ feet under water.
\item A company lost \$$2$ million dollars last year.
\end{itemize}

Each of the above situations can be modeled by a negative number. For example, if today's temperature is $10$ degrees below zero, we say the temperature is $-10$ degrees.

We use the word \textit{integers} to represent the set $\{...,-4,-3,-2,-1,0,1,2,3,4,...\}$. We use a number line to visualize numbers. For example, \cref{fig:NumberLine1} shows the integer $-1$ on the number line.

\begin{tightcenter}
	\begin{tikzpicture}
		\begin{axis}[
				xmin=-5,xmax=5,
				ymin=-1,ymax=1,
				axis y line=none,
				height =1cm,
				grid=none,
				xtick={-4,-3,...,4},
				xlabel={},
					]
			\addplot+[soldot]coordinates{ (-1,0) };				
		\end{axis}
	\end{tikzpicture}
	\captionof{figure}{$-1$ on the number line}
	\label{fig:NumberLine1}
\end{tightcenter}

On a number line, the right is the positive direction, and the left is the negative direction. A bigger number is always located to the right side of a smaller number. 

Let's quickly review two inequality symbols:
\begin{itemize}
\item The symbol ">" is read as "greater than". For example, $2>1$ is read as "two is greater than one."
\item The symbol "<" is read as "less than". For example, $1<2$ is read as "one is less than two."
\end{itemize}

Look at the number line in \cref{fig:NumberLine2} with four numbers marked:

\begin{figure}[!htb]
\centering
	\begin{tikzpicture}
		\begin{axis}[
				xmin=-5,xmax=5,
				ymin=-1,ymax=1,
				axis y line=none,
				height =1cm,
				grid=none,
				xtick={-4,-3,...,4},
				xlabel={},
					]
			\addplot+[soldot]coordinates{ (-1,0) };		
			\addplot+[soldot]coordinates{ (-3,0) };	
			\addplot+[soldot]coordinates{ (0,0) };	
			\addplot+[soldot]coordinates{ (2,0) };			
		\end{axis}
	\end{tikzpicture}
	\captionof{figure}{$-3$, $-1$, $0$ and $2$ on the number line}
	\label{fig:NumberLine2}
\end{figure}

We can observe the following relationship:
\begin{itemize}
\item $2>-3$ since $2$ is located to the right of $-3$;
\item $0>-1$ since $0$ is located to the right of $-1$;
\item $-1>-3$ since $-1$ is located to the right of $-3$.
\end{itemize}

Putting together those four numbers, we have $2>0>-1>-3$. Notice that:
\begin{itemize}
\item Positive numbers are bigger than 0 and negative numbers.
\item The number 0 is bigger than negative numbers.
\item Compare $3>1$ and $-1>-3$. This is because, on the number line, $3$ is located to the right of $1$, and $-1$ is located to the right of $-3$.
\end{itemize}

\subsection{Absolute Value}
The absolute value of a number is the distance between the number and $0$ on the number line.

How far is the number $2$ from $0$ on the number line? The distance is obviously two units, so the absolute value of $2$ is simply $2$.

Similarly, the distance between $-2$ and $0$ on the number line is also two units, so the absolute value of $-2$ is $2$.

\begin{figure}[!htb]
\centering
	\begin{tikzpicture}
		\begin{axis}[
				xmin=-5,xmax=5,
				ymin=-1,ymax=2,
				axis y line=none,
				height =1cm,
				grid=none,
				xtick={-4,-3,...,4},
				xlabel={},
					]	
			\addplot+[soldot]coordinates{ (-2,0) };	
			\addplot[<->,line width=3pt,red,domain=-2:0]{0.5} node[pos=0.5,anchor=south]{2};
			\addplot+[soldot,color=blue]coordinates{ (2,0) };	
			\addplot[<->,line width=3pt,blue,domain=0:2]{0.5} node[pos=0.5,anchor=south]{2};
		\end{axis}

	\end{tikzpicture}
	\captionof{figure}{Absolute Value of $2$ and $-2$}
	\label{fig:AbsoluteValue}
\end{figure}

Here is how we write absolute value with its math symbol:
\begin{itemize}
\item The absolute value of $2$ is $2$, written as $\lvert 2 \rvert =2$.
\item The absolute value of $-2$ is $2$, written as $\lvert -2 \rvert=2$.
\end{itemize}

Here are a few examples:

\[
\begin{aligned}[t]
	\lvert 0 \rvert &=0 \\
	\lvert -1 \rvert &=1 \\
	-\lvert 1 \rvert &=-1
\end{aligned}
\]

The absolute value of $0$ is simply $0$, because the distance between $0$ and $0$ on the number line is $0$.

Note the difference between the last two examples. Absolute value can change a negative number to positive, as long as the number is inside the absolute value symbol, as in $\lvert -1 \rvert =1$.

However, absolute value cannot affect the negative symbol outside the absolute value symbol, as in $-\lvert 1 \rvert =-1$.

