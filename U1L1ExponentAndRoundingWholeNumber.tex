\chapter{Number Basics}
\section{Exponent and Rounding Whole Numbers}


In this lesson, we will learn the concept of exponent, and then learn how to round numbers.

\subsection{Exponent}
Note the difference between these two equations:
\begin{itemize}
\item $2\cdot 3=6$
\item $2^{3}=8$
\end{itemize}

In $2^{3}$, the number $3$ is the exponent. The expression $2^{3}$ means: The number $2$ multiplies itself $3$ times, so we have:
\[ 2^{3}=2 \cdot 2 \cdot 2 = 8 \]

We read $2^{2}$ as "two to the second power", or "two squared".\\*
We read $2^{3}$ as "two to the third power", or "two cubed".\\*
We read $2^{4}$ as "two to the fourth power". Only the second power (squared) and third power (cubed) have special names, because they are regularly used.

Let's look at a few more examples.

\begin{myexample}
\begin{itemize}
\item $3^{2}=3\cdot 3=9$
\item $2^{5}=2\cdot 2\cdot 2\cdot 2\cdot 2=32$
\item $1^{100}=1\cdot 1\cdot 1 \cdot ... \cdot 1=1$
\item $0^{1000}=0\cdot 0\cdot 0 \cdot ... \cdot 0=0$
\end{itemize}
\end{myexample}

From the last two examples, we can see $0$ raised to any power is still $0$, and $1$ raised to any power is still $1$.

\begin{myexample}
\begin{itemize}
\item $2^{1}=2$
\item $3^{1}=3$
\end{itemize}
\end{myexample}

Any number to the first power is simply the number itself. We don't write "to the first power," as in $2^{1}$, except in special cases.

\subsection{Place value}
The number $1,234,567$ is read: one million, two hundred thirty-four thousand, five hundred sixty-seven. We need to learn the name of each digit:
\begin{itemize}
\item $1$ is in the millions place;
\item $2$ is in the hundred thousands place;
\item $3$ is in the ten thousands place;
\item $4$ is in the thousands place;
\item $5$ is in the hundreds place;
\item $6$ is in the tens place;
\item $7$ is in the units place.
\end{itemize}

You need to memorize the name of each place.

\subsection{Rounding}
Why do we need rounding? Assume Portland has $1,987,654$ residents. When we talk about Portland's population, it's silly to say the exact number. Usually we would say Portland has approximately $2$ million people. We rounded the number $1,987,654$ to $2,000,000$. Let me explain rounding with another example.

Assume you only have \$$10.00$ bills. You will purchase a product marked at \$$11.00$. Is it fair to pay \$$10.00$ or \$$20.00$?

Since you only have \$$10.00$ bills, it's fair to pay \$$10.00$, because the marked price \$$11.00$ is closer to \$$10.00$ than to \$$20.00$.

Similarly, if a product is marked at \$$19.00$, it's fair to pay \$$20.00$, because the marked price \$$19.00$ is closer to \$$20.00$ than to \$$10.00$.

What if the product is marked at \$$14.00$ or \$$15.00$? We will explain  rounding rules with examples:

\begin{itemize}
\item To round $10,11,12,13,\text{ or }14$ to the tens place, the answer is $10$.
\item To round $15,16,17,18,\text{ or }19$ to the tens place, the answer is $20$.
\end{itemize}

So, if you only have \$$10.00$ bills, to purchase a product marked at \$$14.00$, it's fair to pay \$$10.00$.

If you only have \$$10.00$ bills, to purchase a product marked at \$$15.00$, it's fair to pay \$$20.00$.

We will summarize rounding rules with an example. We will round $1,234$ to the hundreds place.
\begin{enumerate}
\item Identify the place to be rounded to. It is $2$ in $1,234$.
\item Since we will round to the hundreds place, we either round up to $1,300$, or round to $1,200$.
\item Look at the digit after $2$. In this example, we look at the digit $3$.
	\begin{enumerate}
		\item If this digit is $0,1,2,3\text{ or }4$, we don't round up. In this example, $1,234$ is rounded to $1,200$.
		\item If this digit is $5,6,7,8\text{ or }9$, we round up.	
	\end{enumerate}
\end{enumerate}

Again, why do we need to round $1,234$ to $1,200$? Assume you purchased a used car for \$$1,234$. If a friend asks you how much the car cost, you would most likely say it cost about \$$1,200$, instead of saying the exact number. You would assume your friend don't care about those extra \$$34.00$.

Let's look at some more examples.

\begin{myexample}
Round $1,234,567$ to the thousands place.
\end{myexample}
\begin{solution}
\begin{enumerate}
\item Identify the place to be rounded to. The thousands place in $1,234,567$ is $4$.
\item Since we will round to the thousands place, we either round up to $1,235,000$, or round to $1,234,000$.
\item Look at the digit after the thousands place. In this example, we look at the digit $5$. By rounding rules, $1,234,567$ is rounded up to $1,235,000$.
\end{enumerate}
\end{solution}

Be careful when we round up $9$. Look at the following examples.

\begin{myexample}
Round $1,961$ to the hundreds place.
\end{myexample}
\begin{solution}
\begin{enumerate}
\item Identify the place to be rounded to. The hundreds place in $1,961$ is $9$.
\item Since we will round to the hundreds place, we either round up to $2,000$, or round to $1,900$.
\item Look at the digit after the hundreds place. In this example, we look at the digit $6$. By rounding rules, $1,961$ is rounded up to $2,000$.
\end{enumerate}
\end{solution}

\begin{myexample}
Round $1,995$ to the tens place.
\end{myexample}
\begin{solution}
\begin{enumerate}
\item Identify the place to be rounded to. The tens place in $1,995$ is $9$ (the $9$ in front of $5$).
\item Since we will round to the tens place, we either round up to $2,000$, or round to $1,990$.
\item Look at the digit after the tens place. In this example, we look at the digit $5$. By rounding rules, $1,995$ is rounded up to $2,000$.
\end{enumerate}
\end{solution}
