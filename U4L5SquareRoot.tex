
\section{Square Root}

\subsection{Definition of Square Root}
Let's review the definition of square:
\[
\begin{aligned}[t]
	0^{2} &= 0\cdot0 = 0 \\
	1^{2} &= 1\cdot1 = 1 \\
	2^{2} &= 2\cdot2 = 4 \\
	3^{2} &= 3\cdot3 = 9 \\
	9^{2} &= 9\cdot9 = 81
\end{aligned}
\]

Again, it's critical to memorize the following square numbers: \[0,1,4,9,16,25,36,49,64,81,100,121,144\]

Square root is the inverse operation of square. For example, if we want to know which number squared gives the number $4$, we write $\sqrt{4}$. Since $2^{2}=4$, we have $\sqrt{4}=2$. Here are a few more examples:
\[
\begin{aligned}[t]
	\sqrt{0} &= 0 && \text{as } 0^{2}=0 \\
	\sqrt{1} &= 1 && \text{as } 1^{2}=1 \\
	\sqrt{4} &= 2 && \text{as } 2^{2}=4 \\
	\sqrt{9} &= 3 && \text{as } 3^{2}=9 \\
	\sqrt{81} &= 9 && \text{as } 9^{2}=81
\end{aligned}
\]

Most of the time, the square root of an integer is an irrational decimal. We use calculators to find the square root of such numbers:
\[
\begin{aligned}[t]
	\sqrt{2} &= 1.414...\\
	\sqrt{3} &= 1.732...\\
	\sqrt{1000} &= 31.622...
\end{aligned}
\]

\subsection{Square Root of Fractions Involving Perfect Squares}
Let's look at a few examples:
\[
\begin{aligned}[t]
	\sqrt{\frac{1}{4}} &= \frac{1}{2} && \text{as } \left( \frac{1}{2} \right)^{2}=\frac{1}{2} \cdot \frac{1}{2} = \frac{1}{4} \\
	\sqrt{\frac{1}{9}} &= \frac{1}{3} && \text{as } \left(\frac{1}{3}\right)^{2}=\frac{1}{3} \cdot \frac{1}{3} = \frac{1}{9} \\
	\sqrt{\frac{4}{9}} &= \frac{2}{3} && \text{as } \left(\frac{2}{3}\right)^{2}=\frac{2}{3} \cdot \frac{2}{3} = \frac{4}{9} \\
\end{aligned}
\]

To calculate the square root of a fraction, like $\sqrt{\frac{4}{9}}$, we need to take the square root of both the numerator and denominator:
\[ \sqrt{\frac{4}{9}} = \frac{\sqrt{4}}{\sqrt{9}} = \frac{2}{3} \]

\subsection{Square Root of Decimals Involving Square Numbers}
Let's review something we learned earlier:
\begin{itemize}
\item When we calculate $0.2\cdot0.2$, first we do $2\cdot2=4$. From $2\cdot2$ to $0.2\cdot0.2$, the decimal point moved to the left twice in total, so we move the decimal point of $4$ to the left twice and have $0.2\cdot0.2=0.04$.
\item When we calculate $0.11\cdot0.11$, first we do $11\cdot11=121$. From $11\cdot11$ to $0.11\cdot0.11$, the decimal point moved to the left four times in total, so we move the decimal point of $121$ to the left four times and have $0.11\cdot0.11=0.0121$.
\end{itemize}

Now let's look at a few examples of square root involving decimals:
\[
\begin{aligned}[t]
	\sqrt{0.04} &= 0.2 && \text{as } 0.2^{2}=0.2 \cdot 0.2=0.04 \\
	\sqrt{1.14} &= 1.2 && \text{as } 1.2^{2}=1.2 \cdot 1.2=1.14 \\
	\sqrt{0.0004} &= 0.02 && \text{as } 0.02^{2}=0.02 \cdot 0.02=0.0004
\end{aligned}
\]

We could summarize a rule here. However, it's better to jot down a few numbers on scratch paper when calculating square root of decimals. For example, to calculate $\sqrt{0.0081}$, recognize that $9^{2}=81$, so we know the the answer could be $0.9$, $0.09$ or $0.009$. Since $0.09^{2}=0.09\cdot0.09=0.0081$, we know $\sqrt{0.0081}=0.09$

\subsection{Square Root of Other Decimals}
Most of the time, the square root of a decimal is an irrational decimal. We use calculators to find the square root of such numbers:
\[
\begin{aligned}[t]
	\sqrt{12.1} &= 3.4785...\\
	\sqrt{0.1} &= 0.3162...\\
	\sqrt{0.4} &= 0.6324...
\end{aligned}
\]

Compare the square root of these two numbers:
\[
\begin{aligned}[t]
	\sqrt{0.04} &= 0.2 \\
	\sqrt{0.4} &= 0.6324...
\end{aligned}
\]

\subsection{Square Root of Negative Numbers}
When we evaluate $\sqrt{9}$, we are looking for a number whose square is $9$. Since $3^{2}=9$, we have $\sqrt{9}=3$.

How about $\sqrt{-9}$? We are looking for a number whose square is $-9$. Well, let's try $-3$: We have $(-3)^{2}=(-3)(-3)=9$, so $-3$ is not the square root of $-9$.

We cannot find such a number, because the square of any negative number is positive! Since we cannot find the square root of $-9$, we say $\sqrt{-9}$ doesn't exist, or \textit{undefined}.

