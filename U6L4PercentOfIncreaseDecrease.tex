
\section{Percent of Increase/Decrease}

Occasionally we read news like the following:

\begin{itemize}
\item Cisco posted a good earning season and its stock increased by $2.5\%$ yesterday.
\item Since the recession, the population in this town has decreased by about $30\%$.
\end{itemize}

In this lesson, we will learn percent of increase/decrease. Here is the key: For percent of increase/decrease, we are talking about the increase/decrease with respect to the \textit{original} value.

\subsection{Calculate Percent of Increase/Decrease}

\begin{myexample}
Mary just got a pay raise from \$$15.00$ per hour to \$$15.75$ per hour. What was the percent of increase?
\end{myexample}
\begin{solution}

\textbf{Method 1:} First, we find the amount of increase by subtraction: \$$15.75-$\$$15.00=$\$$0.75$.

Next, we need to find \$$0.75$ is what percent of the \textit{original} value---\$$15.00$. This is a Type II percent problem. We will use the Percent Formula to solve this problem.

Assume $0.75$ is $x$ (as a percent) of $15$. We will write down the "Percent Formula" and the problem right next to each other:
\[
\begin{aligned}[t]
	&3 &= &&50\% &&\cdot &&6 \\
	&0.75 &= &&x \text{ (as a percent)} &&\cdot &&15
\end{aligned}
\]

Next, we can solve for $x$ in the equation:
\[
\begin{aligned}[t]
	0.75 &= x \cdot 15 \\
	0.75 &= 15x \\
	\frac{0.75}{15} &= \frac{15x}{15} \\
	0.05 &= x \\
	5\% &= x
\end{aligned}
\]
\textbf{Conclusion: } Mary got a $5\%$ pay raise.

\textbf{Method 2: } First, we find the new pay rate, \$$15.75$, is what percent of the old pay rate, \$$15.00$. This is a Type II percent problem. Assume $15.75$ is $x$ (as a percent) of $15$. We can solve the following equation:
\[
\begin{aligned}[t]
	15.75 &= x \cdot 15 \\
	15.75 &= 15x \\
	\frac{15.75}{15} &= \frac{15x}{15} \\
	1.05 &= x \\
	105\% &= x
\end{aligned}
\]
The new pay rate is $105\%$ of the old pay rate, implying the percent of increase is $105\%-100\%=5\%$.

\textbf{Conclusion: } Mary got a $5\%$ pay raise.

\end{solution}

\begin{myexample}
Since the recession, a town's population decreased from $279$ to $221$. What's the percent of decrease? Round your percent to a whole number.
\end{myexample}
\begin{solution}

\textbf{Method 1:} First, we find the amount of decrease by subtraction: $279-221=58$.

Next, we need to find $58$ is what percent of the \textit{original} value---$279$. This is a Type II percent problem. We will use multiplication/division to solve this problem.

No variable ($x$) is involved in this method. The key is to write down a simple example on scratch paper, and then put numbers in their corresponding places.

To find "$3$ is what percent of $6$", we do:
\[ 3\div6=0.5=50\% \]
Similarly, to find "$58$ is what percent of $279$", we do:
\[ 58\div279 \approx 0.21 \approx 21\% \]

\textbf{Conclusion: } The town's population decreased by approximately $21\%$.

\textbf{Method 2: } First, we find $221$ is what percent of the \textit{original} value---$279$. This is a Type II percent problem. We have:
\[ 221\div279 \approx 0.79 \approx 79\% \]

Since the new value is $79\%$ of the original value, the percent of decrease is $100\%-79\%=21\%$.

\textbf{Conclusion: } The town's population decreased by approximately $21\%$.

\end{solution}

Sometimes the increase is over $100\%$.

\begin{myexample}
Mary used to make \$$12.00$ per hour. After she earned a Bachelor's degree, she found a new job which pays \$$33.00$ per hour. What was the percent of increase in her pay?
\end{myexample}
\begin{solution}

\textbf{Method 1:} First, we find the amount of increase by subtraction: \$$33.00-$\$$12.00=$\$$21.00$.

Next, we need to find \$$21.00$ is what percent of the \textit{original} value---\$$12.00$. This is a Type II percent problem. We will use the Percent Formula to solve this problem.

Assume $21$ is $x$ (as a percent) of $12$. We will write down the "Percent Formula" and the problem right next to each other:
\[
\begin{aligned}[t]
	&3 &= &&50\% &&\cdot &&6 \\
	&21 &= &&x \text{ (as a percent)} &&\cdot &&12
\end{aligned}
\]

Next, we solve for $x$ in the equation:
\[
\begin{aligned}[t]
	21 &= x \cdot 12 \\
	21 &= 12x \\
	\frac{21}{12} &= \frac{12x}{12} \\
	1.75 &= x \\
	175\% &= x
\end{aligned}
\]
\textbf{Conclusion: } The increase in Mary's pay rate was $175\%$.

\textbf{Method 2: } First, we find the new pay rate, \$$33.00$, is what percent of the old pay rate, \$$12.00$. This is a Type II percent problem. Assume $33$ is $x$ (as a percent) of $12$. We solve the following equation:
\[
\begin{aligned}[t]
	33 &= x \cdot 12 \\
	33 &= 12x \\
	\frac{33}{12} &= \frac{12x}{12} \\
	2.75 &= x \\
	275\% &= x
\end{aligned}
\]
The new pay rate is $275\%$ of the old pay rate, implying the percent of increase is $275\%-100\%=175\%$.

\textbf{Conclusion: } The increase in Mary's pay rate was $175\%$.

\end{solution}

\subsection{Increase and Decrease in Succession}
If a value increased and then decreased by the same percentage, the result is often counter-intuitive.

\begin{myexample}
A house was purchased for \$$200,000.00$. Last year, the house's value increased $5\%$, and then decreased $5\%$. What's the house's current value after the changes?
\end{myexample}
\begin{solution}
Intuitively, the house's value didn't change, but intuition doesn't work sometimes.

First, the house's value increased $5\%$ from \$$200,000.00$. To find $5\%$ of \$$200,000.00$, we do:
\[ 5\% \cdot 200000 = 0.05 \cdot 200000 = 10000 \]
After the increase, the house's value became \$$200,000+$\$$10,000=$\$$210,000$.

Next, the house's value decreased $5\%$ from \$$210,000.00$. To find $5\%$ of \$$210,000.00$, we do:
\[ 5\% \cdot 210000 = 0.05 \cdot 210000 = 10500 \]
After the decrease, the house's value becomes \$$210,000-$\$$10,500=$\$$199,500$.

\textbf{Conclusion:} The house's current value after the changes is \$$199,500$.

The $5\%$ decrease is more than the $5\%$ increase, because the $5\%$ decrease was with respect to a bigger value (after the increase).
\end{solution}

\subsection{More Challenging Percent of Increase/Decrease Problems}
The key to do percent of increase/decrease questions is to think of this question: The new value is what percent of the original value?

\begin{myexample}
Your favorite sweater is on sale! With $30\%$ markdown, the new price is \$$49.00$. What was the sweater's regular price (before the markdown)?
\end{myexample}
\begin{solution}
After the $30\%$ markdown, the new price is $70\%$ of the original price. Now the question becomes: \$$49.00$ is $70\%$ of what? This is a Type III percent problem. We will use the percent formula to solve this problem.

Assume $49$ is $70\%$ of $x$. We will write down the "Percent Formula" and the problem right next to each other:
\[
\begin{aligned}[t]
	&3 &= &&50\% &&\cdot &&6 \\
	&49 &= &&70\% &&\cdot &&x
\end{aligned}
\]

Next, we can solve for $x$ in the equation:
\[
\begin{aligned}[t]
	49 &= 70\% \cdot x \\
	49 &= 0.7x \\
	\frac{49}{0.7} &= \frac{0.7x}{0.7} \\
	70 &= x
\end{aligned}
\]
\textbf{Conclusion: } The sweater's regular price (before the markdown) is \$$70.00$.

\end{solution}

\begin{myexample}
A sweater's price was marked up by $30\%$. After the markup, its new price is \$$78.00$. What was the sweater's price before the markup?
\end{myexample}
\begin{solution}
After the $30\%$ price markup, the new price is $130\%$ of the original price. Now the question becomes: \$$78.00$ is $130\%$ of what? This is a Type III percent problem. We will use the percent formula to solve this problem.

Assume $78$ is $130\%$ of $x$. We will write down the "Percent Formula" and the problem right next to each other:
\[
\begin{aligned}[t]
	&3 &= &&50\% &&\cdot &&6 \\
	&78 &= &&130\% &&\cdot &&x
\end{aligned}
\]

Next, we solve for $x$ in the equation:
\[
\begin{aligned}[t]
	78 &= 130\% \cdot x \\
	78 &= 1.3x \\
	\frac{78}{1.3} &= \frac{1.3x}{1.3} \\
	60 &= x
\end{aligned}
\]
\textbf{Conclusion: } The sweater's price was \$$60.00$ before the markup.

\end{solution}

