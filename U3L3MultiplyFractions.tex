
\section{Multiply Fractions}

In this lesson, we will learn how to multiply fractions.

\subsection{Multiply Two Fractions}

We will start with a very important concept: The English word "of," in many cases, can be translated into the multiplication sign in math. For example, "twice of $5$" can be translated into $2\cdot5$.

Similarly, "half of half a dollar (50 cents)" can be translated into $\frac{1}{2}\cdot\frac{1}{2}$. We know the answer should be 25 cents, or $\frac{1}{4}$. We have:
\[ \frac{1}{2}\cdot\frac{1}{2}=\frac{1}{4} \]

We can see how to multiply fractions: We simply multiply the numerators and denominators. This is much easier than adding fractions---There is no need to find the common denominator when we multiply fractions.

\begin{myexample}
\[ 
\begin{aligned}[t]
	&\phantom{{}=}\frac{2}{3} \cdot \frac{2}{5} \\
	&= \frac{2\cdot2}{3\cdot5} \\
	&= \frac{4}{15}
\end{aligned}
\]
\end{myexample}

Remember: We must reduce fraction if we can. See the next example.

\begin{myexample}
\[ 
\begin{aligned}[t]
	&\phantom{{}=}\frac{2}{3} \cdot \frac{3}{5} \\
	&= \frac{2\cdot3}{3\cdot5} \\
	&= \frac{6}{15} \\
	&= \frac{6\div3}{15\div3} \\
	&= \frac{2}{5}
\end{aligned}
\]
In the example above, we could reduce fractions before we multiply across:
\[ 
\begin{aligned}[t]
	&\phantom{{}=}\frac{2}{3} \cdot \frac{3}{5} \\
	&= \frac{2}{3\div3} \cdot \frac{3\div3}{5} \\
	&= \frac{2}{1} \cdot \frac{1}{5} \\
	&= \frac{2\cdot1}{1\cdot5} \\
	&= \frac{2}{5}
\end{aligned}
\]
This skill will save us a lot of time if numbers are big. See the next example.
\end{myexample}

\begin{myexample}
\[ 
\begin{aligned}[t]
	&\phantom{{}=}\frac{4}{5} \cdot \frac{3}{8} \cdot \frac{5}{9} \\
	&= \frac{4{\color{red}\div4}}{5\color{blue}\div5} \cdot \frac{3\color{brown}\div3}{8\color{red}\div4} \cdot \frac{5\color{blue}\div5}{9\color{brown}\div3} \\
	&= \frac{1}{1} \cdot \frac{1}{2} \cdot \frac{1}{3} \\
	&= \frac{1\cdot1\cdot1}{1\cdot2\cdot3} \\
	&= \frac{1}{6}
\end{aligned}
\]
In this example, if we don't reduce fractions first, we have to deal with big numbers like $4\cdot3\cdot5=60$ and $5\cdot8\cdot9=360$.
\end{myexample}

Here is a common mistake:
\[
\begin{aligned}[t]
	&\phantom{{}=}\frac{2}{3}\cdot\frac{2}{5} \\
	&=\frac{2\div2}{3}\cdot\frac{2\div2}{5} \\
	&=\frac{1}{3}\cdot\frac{1}{5} \\
	&=\frac{1}{15}
\end{aligned}
\]
This is incorrect. When we reduce a fraction, we must divide a number in both the numerator and denominator. We cannot divide a number in two numerators, as in the example above. The correct answer is $\frac{2}{3}\cdot\frac{2}{5}=\frac{4}{15}$.

\subsection{Multiply a Fraction and an Integer}
When we multiply a fraction and an integer, we need to change the integer to a fraction, and then multiply two fractions.

\begin{myexample}
\[ 
\begin{aligned}[t]
	&\phantom{{}=}\frac{2}{9} \cdot 2 \\
	&= \frac{2}{9} \cdot \frac{2}{1} \\
	&= \frac{2\cdot2}{9\cdot1} \\
	&= \frac{4}{9}
\end{aligned}
\]
We can change $2$ into $\frac{2}{1}$ because $2\div1=2$.
\label{ex:MultiplyFraction1}
\end{myexample}

We need to learn an important shortcut, which will save you tons of time later. Recall that the fraction line is the same as the division symbol. We can do:
\[ \frac{3}{5}\cdot{10} = 10\div5\cdot3 = 2\cdot3=6 \]
Let's look at a few more examples:
\[
\begin{aligned}[t]
	&\frac{2}{3}\cdot{3}=3\div3\cdot2=1\cdot2=2 \\
	&\frac{2}{3}\cdot{6}=6\div3\cdot2=2\cdot2=4 \\
	&\frac{2}{3}\cdot{9}=9\div3\cdot2=3\cdot2=6 \\
\end{aligned}
\]

This shortcut works as long as the denominator goes into the integer evenly. If not, we have to change the integer to a fraction, and then do fraction multiplication like in \cref{ex:MultiplyFraction1}.

\subsection{Fraction Multiplication Word Problems}
\begin{myexample}
A school won a \$$5,000$ grant, and will use $\frac{3}{4}$ of the grant to purchase graphing calculators. How much money will be used to purchase graphing calculators?
\end{myexample}
\begin{solution}
This problem can be boiled down to this question: What is $\frac{3}{4}$ of $5,000$? Again, the word "of" can be translated into the multiplication sign in math. We have:
\[ 
\begin{aligned}[t]
	&\phantom{{}=}\frac{3}{4} \cdot 5000 \\
	&= 5000\div4\cdot3 \\
	&= 1250\cdot3 \\
	&= 3750
\end{aligned}
\]
We can use the shortcut to do fraction multiplication because the denominator $4$ goes into $5000$ evenly.

\textbf{Conclusion:} The school will spend \$$3,750$ to purchase graphing calculators.
\end{solution}

\begin{myexample}
A school won a grant, and will evenly share the grant among $8$ classes. In one class, Mr. Smith will use $\frac{2}{3}$ of the money for the class to purchase books. Mr. Smith will use what fraction of the school's grant to purchase books?
\end{myexample}
\begin{solution}
Since the grant is evenly shared by $8$ classes, each class gets $\frac{1}{8}$ of the grant.

Now this problem can be boiled down to this question: What is $\frac{2}{3}$ of $\frac{1}{8}$? Again, the word "of" can be translated into the multiplication sign in math. We have:
\[ 
\begin{aligned}[t]
	&\phantom{{}=}\frac{2}{3} \cdot \frac{1}{8} \\
	&= \frac{2\div2}{3} \cdot \frac{1}{8\div2} \\
	&= \frac{1}{3} \cdot \frac{1}{4} \\
	&= \frac{1\cdot1}{3\cdot4} \\
	&= \frac{1}{12}
\end{aligned}
\]
\textbf{Conclusion:} Mr. Smith will use $\frac{1}{12}$ of the school's grant to purchase books.
\end{solution}

\subsection{Summary}
Let's review what we learned in this lesson:
\begin{itemize}
\item To multiply two fractions, we simply multiply the numerators and  denominators. For example: 
\[ \frac{1}{2} \cdot \frac{1}{3}=\frac{1\cdot1}{2\cdot3}=\frac{1}{6} \]
\item When we multiply fractions, if we can reduce fractions before multiplying the numerators and denominators, we should reduce fractions first. This will avoid dealing with big numbers. For example:
\[ \frac{4}{7} \cdot \frac{3}{8} = \frac{4\div4}{7} \cdot \frac{3}{8\div4} = \frac{1}{7} \cdot \frac{3}{2} = \frac{3}{14} \]
\item To multiply a fraction and an integer, we first change the integer to a fraction, and then multiply the numerators and denominators. For example:
\[ \frac{1}{3} \cdot 2 = \frac{1}{3} \cdot \frac{2}{1} = \frac{2}{3} \]
\item When we multiply a fraction and an integer, if the denominator can go into the integer evenly, there is a shortcut:
\[ \frac{2}{3} \cdot 6 = 6 \div 3 \cdot 2 = 2 \cdot 2 = 4 \]
\end{itemize}

