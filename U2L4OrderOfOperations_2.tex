
\section{Order of Operations Involving Negative Numbers}
\thispagestyle{fancy}

Earlier, we learned order of operations--PEMDAS. In this lesson, we introduce negative numbers and absolute value into PEMDAS.

\subsection{Order of Operations with Negative Numbers}

We already learned the basics of order of operations. We will look at a few examples involving negative numbers.

\begin{myexample}
\[
\begin{aligned}[t]
   &\phantom{{}=} 2-(\underline{1-3}) \\
   &= 2-(-2) \\
   &= 2+2 \\
   &= 4
\end{aligned}
\]
In the second step, it would be confusing to write $2--2$. When we write a negative number, it doesn't hurt to use a pair of parentheses to make things clear.
\end{myexample}

\begin{myexample}
\begin{tabular}[t]{c@{\hspace{4cm}}c@{\hspace{2cm}}c}
&
$ \begin{aligned}[t] 
	&\phantom{{}=} 7-2 \\ 
	&= 5
  \end{aligned} $ 
&
$ \begin{aligned}[t] 
	&\phantom{{}=} 7(-2) \\ 
	&= -14
  \end{aligned} $ 
\end{tabular}

The parentheses make a difference! On the right side, since there is no operation symbol between $7$ and $(-2)$, it implies multiplication.
\end{myexample}

\begin{myexample}
\begin{tabular}[t]{c@{\hspace{4cm}}c@{\hspace{2cm}}c}
&
$ \begin{aligned}[t] 
	&\phantom{{}=} 2-\underline{(-1)^{2}} \\ 
	&= 2-1 \\ 
	&= 1 
  \end{aligned} $ 
&
$ \begin{aligned}[t] 
	&\phantom{{}=} 2-\underline{(-1)^{3}} \\ 
	&= 2-(-1) \\ 
	&= 2+1 \\
	&= 3
  \end{aligned} $ 
\end{tabular}

Note the difference between $(-1)^{2}=1$ and $(-1)^{3}=-1$.
\end{myexample}

\begin{myexample}
\begin{tabular}[t]{c@{\hspace{4cm}}c@{\hspace{2cm}}c}
&
$ \begin{aligned}[t] 
	&\phantom{{}=} \underline{(-5)^{2}}+10 \\ 
	&= 25+10 \\ 
	&= 35
  \end{aligned} $ 
&
$ \begin{aligned}[t] 
	&\phantom{{}=} \underline{-5^{2}}+10 \\ 
	&= -25+10 \\ 
	&= -15
  \end{aligned} $ 
\end{tabular}

Note the difference between $(-5)^{2}=25$ and $-5^{2}=-25$.
\end{myexample}

\begin{myexample}
Evaluate $2-3(-4)$
\end{myexample}
\begin{solution}
First of all, notice that there is no symbol between $3$ and $(-4)$, which implies multiplication.

Since multiplication overrides subtraction, we cannot do $2-3$ before doing $3\cdot(-4)$.

Earlier, we learned that the subtraction symbol has the same function as the negative symbol, as in $3-1=2$ and $3+(-1)=2$. There are two ways to do this problem.

\begin{tabular}[t]{c@{\hspace{4cm}}c@{\hspace{2cm}}c}
&
$ \begin{aligned}[t] 
	&\phantom{{}=} 2-3(-4) \\ 
	&= 2-(-12) \\ 
	&= 2+12 \\
	&= 14
  \end{aligned} $ 
&
$ \begin{aligned}[t] 
	&\phantom{{}=} 2-3(-4) \\ 
	&= 2+(-3)(-4) \\ 
	&= 2+12 \\
	&= 14
  \end{aligned} $ 
\end{tabular}

On the left side, we treated the subtraction sign simply as a subtraction sign, and copied it to the next step.

On the right side, we treated the subtraction sign as a negative sign.
\end{solution}

\subsection{Order of Operations involving Absolute Value}

We need to modify the acronym PEMDAS to include absolute values: 

\[
\begin{aligned}[t]
   &P &\text{(Parentheses \textit{and Absolute Value})} \\
   &E &\text{(Exponent)} \\
   &MD &\text{(Multiplication and Division)} \\
   &AS &\text{(Addition and Subtraction)}
\end{aligned}
\]
\captionof{figure}{Order of Operations}

Basically, the absolute value symbols have the same priority as parentheses, except it changes negative numbers to positive. Let's look at a few examples.

\begin{myexample}
\begin{tabular}[t]{c@{\hspace{4cm}}c@{\hspace{2cm}}c}
&
$ \begin{aligned}[t] 
	&\phantom{{}=} 3-|\underline{1-4}| \\ 
	&= 3-\underline{|-3|} \\ 
	&= 3-3 \\
	&= 0
  \end{aligned} $ 
&
$ \begin{aligned}[t] 
	&\phantom{{}=} 3-(\underline{1-4}) \\ 
	&= 3-(-3) \\ 
	&= 3+3 \\
	&= 6
  \end{aligned} $ 
\end{tabular}

Note the difference between $|1-4|=3$ and $(1-4)=-3$.

It's common mistake to change $|1-4|$ to $|1+4|$. We must complete all operations inside the absolute value symbols before changing the number inside to positive.
\end{myexample}

\begin{myexample}
\[
\begin{aligned}[t]
	&\phantom{{}=} 3-2|\underline{4-1}| \\ 
	&= 3-2\underline{|3|} \\ 
	&= 3-\underline{2(3)} \\
	&= 3-6 \\
	&= -3
\end{aligned}
\]
It's a common mistake to do $3-2$ first. Notice the implied multiplication between $2$ and $|4-1|$. We must do multiplication before doing subtraction.

Also, from $2|3|$ to $2(3)$, notice we added a pair of parentheses around $3$. Otherwise, we must add a dot, $2\cdot3$, to make it clear the operation is multiplication. We just cannot write $23$.
\end{myexample}

\begin{myexample}
\begin{tabular}[t]{c@{\hspace{4cm}}c@{\hspace{2cm}}c}
&
$ \begin{aligned}[t] 
	&\phantom{{}=} \frac{|1-3|}{-1} \\ 
	&= \frac{|-2|}{-1} \\ 
	&= \frac{2}{-1} \\
	&= -2
  \end{aligned} $ 
&
$ \begin{aligned}[t] 
	&\phantom{{}=} \left| \frac{1-3}{-1} \right| \\ 
	&= \left| \frac{-2}{-1} \right|  \\ 
	&= \left| 2 \right|  \\
	&= 2
  \end{aligned} $ 
\end{tabular}

Note that the length of the absolute value symbols make a difference!
\end{myexample}

