
\section{Proportion}

\subsection{Motivation of Using Proportion}
Let's start by doing a problem with rate.

\begin{myexample}
A car drove $150$ miles in $6$ hours. How long would it take the car to drive $250$ miles?
\end{myexample}
\begin{solution}
First, we use division to find the rate of change (speed in this case):
\[ \frac{150 \text{ miles}}{6 \text{ hours}}=25 \text{ miles/hour} \]
The car's speed is $25$ miles per hour.

Next, we find how long it would take the car to drive $250$ miles by fraction multiplication. We will use the rate $\frac{1 \text{ hour}}{25 \text{ miles}}$, and we have:
\[
\begin{aligned}[t]
	&\phantom{{}=}250 \text{ miles} \cdot \frac{1 \text{ hour}}{25 \text{ miles}} \\
	&= \frac{250 \text{ miles}}{1} \cdot \frac{1 \text{ hour}}{25 \text{ miles}} \\
	&= \frac{250}{1} \cdot \frac{1 \text{ hour}}{25} \\
	&= \frac{250 \cdot 1}{1 \cdot 25} \text{ hours} \\
	&= \frac{250}{25} \text{ hours} \\
	&= 10 \text{ hours}
\end{aligned}
\]

It takes the car $10$ hours to drive $250$ miles.
\end{solution}

With proportion, solving this problem becomes much easier. The core of the solution is below (we will learn details later in this lesson):
\[
\begin{aligned}[t]
	\frac{150 \text{ miles}}{6 \text{ hours}} &= \frac{250 \text{ miles}}{x \text{ hours}} \\
	150x &= 250 \cdot 6 \\
	150x &= 1500 \\
	\frac{150x}{150} &= \frac{1500}{150} \\
	x &= 10
\end{aligned}
\]

We will learn how to set up proportion and then solve it. We start by learning how to solve equations like $150x=1500$.

\subsection{Solve Simple Equations}
Think about this puzzle: $2$ times which number gives $10$? If we use $x$ to represent the unknown number, we can write an equation:
\[ 2x=10 \]

We can omit the multiplication symbol between $2$ and $x$, as $2x$ implies $2$ times $x$.

We know the answer is $5$. How do we solve this equation mathematically? Since division is the inverse operation of multiplication, we divide both sides of the equation with $2$:
\[
\begin{aligned}[t]
	\phantom{{}=}2x &=10 \\
	\frac{2x}{2} &= \frac{10}{2} \\
	x &= 5
\end{aligned}
\]

On the left side, from $\frac{2x}{2}$, since $2\div2=1$, we have $1x$. Since $1$ times any number will not change that number's value (for example, $1\cdot3=3,1\cdot4=4,...$), $1x$ is the same as $x$.

Here are two more examples:

\begin{tabular}[t]{c@{\hspace{4cm}}c@{\hspace{2cm}}c}
&
$ \begin{aligned}[t] 
	\phantom{{}=} 3x &= 12 \\ 
	\frac{3x}{3} &= \frac{12}{3} \\ 
	x &= 4
  \end{aligned} $ 
&
$ \begin{aligned}[t] 
	\phantom{{}=} 15x &= 45 \\ 
	\frac{15x}{15} &= \frac{45}{15} \\ 
	x &= 3
  \end{aligned} $ 
\end{tabular}

Basically, to solve an equation like $3x=12$, we divide both sides of the equation by the number in front of $x$.

\subsection{Cross-Multiplication}
Let's observe a pattern:
\[
\begin{aligned}[t]
	&\frac{1}{2}=\frac{3}{6} &\rightarrow &&1\cdot6=2\cdot3 \\
	&\frac{1}{2}=\frac{4}{8} &\rightarrow &&1\cdot8=2\cdot4 \\
	&\frac{3}{6}=\frac{4}{8} &\rightarrow &&3\cdot8=6\cdot4
\end{aligned}
\]

We can see why this pattern is called "cross-multiplication". Now we can solve proportions. Let's look at a few examples:

\begin{tabular}[t]{c@{\hspace{0.5cm}}c@{\hspace{2cm}}c@{\hspace{2cm}}c@{\hspace{2cm}}c}
&
$ \begin{aligned}[t] 
	\frac{x}{6} &= \frac{2}{3} \\
	3x &= 6 \cdot 2 \\ 
	3x &= 12 \\ 
	\frac{3x}{3} &= \frac{12}{3} \\ 
	x &= 4
  \end{aligned} $ 
&
$ \begin{aligned}[t] 
	\frac{4}{x} &= \frac{2}{3} \\
	2x &= 4 \cdot 3 \\ 
	2x &= 12 \\ 
	\frac{2x}{2} &= \frac{12}{2} \\ 
	x &= 6
  \end{aligned} $ 
&
$ \begin{aligned}[t] 
	\frac{2}{3} &= \frac{x}{6} \\
	3x &= 2 \cdot 6 \\ 
	3x &= 12 \\ 
	\frac{3x}{3} &= \frac{12}{3} \\ 
	x &= 4
  \end{aligned} $ 
&
$ \begin{aligned}[t] 
	\frac{2}{3} &= \frac{4}{x} \\
	2x &= 3 \cdot 4 \\ 
	2x &= 12 \\ 
	\frac{2x}{2} &= \frac{12}{2} \\ 
	x &= 6
  \end{aligned} $ 
\end{tabular}

\subsection{Proportion Word Problems}
It's important to organize information in a table when we write proportion equations. Let's look at a few examples.

\begin{myexample}
A car drove $150$ miles in $6$ hours. How long would it take the car to drive $250$ miles?
\end{myexample}
\begin{solution}
First, assume it would take the car $x$ hours to drive $250$ miles. Next, we will use a table to organize the given information:

\begin{center}
\begin{tabular}{ | c | c | c | }
	\hline
    	& \textbf{Situation 1} & \textbf{Situation 2} \\ \hline
  \textbf{miles} & $150$ & $250$ \\ \hline
  \textbf{hours} & $6$ & $x$ \\ \hline
\end{tabular}
\end{center}

Now we can write a proportion equation and solve for $x$. It's critical to include units in the equation to make sure numbers are in the right places.
\[
\begin{aligned}[t]
	\frac{150 \text{ miles}}{6 \text{ hours}} &= \frac{250 \text{ miles}}{x \text{ hours}} \\
	150x &= 250 \cdot 6 \\
	150x &= 1500 \\
	\frac{150x}{150} &= \frac{1500}{150} \\
	x &= 10
\end{aligned}
\]

It would take the car $10$ hours to drive $250$ miles.
\end{solution}

In the example above, if we made a mistake by writing $\frac{150 \text{ miles}}{6 \text{ hours}} = \frac{x \text{ hours}}{250 \text{ miles}}$, it's easy to see the units don't match up, and thus the equation is incorrect. This is why it's important to include units in the equation.

\begin{myexample}
A restaurant's expense of labor cost to food cost is in the ratio of $8:3$. In one month, if the restaurant spent \$$600.00$ in food cost, how much did it spend on labor cost?
\end{myexample}
\begin{solution}
Assume the restaurant spent $x$ dollars on labor in that month. We use a table to organize the given information:

\begin{center}
\begin{tabular}{ | c | c | c | }
	\hline
    	& \textbf{Ratio} & \textbf{In that month} \\ \hline
  \textbf{labor cost in dollars} & $8$ & $x$ \\ \hline
  \textbf{food cost in dollars} & $3$ & $600$ \\ \hline
\end{tabular}
\end{center}

We write a proportion equation and solve for $x$.
\[
\begin{aligned}[t]
	\frac{8 \text{ labor cost in dollars}}{3 \text{ food cost in dollars}} &= \frac{x \text{ labor cost in dollars}}{600 \text{ food cost in dollars}} \\
	3x &= 8\cdot 600 \\
	3x &= 4800 \\
	\frac{3x}{3} &= \frac{4800}{3} \\
	x &= 1600
\end{aligned}
\]

In one month, if the restaurant spent \$$600.00$ in food cost, it spent \$$1,600.00$ on labor cost.
\end{solution}

