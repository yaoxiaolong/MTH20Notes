
\section{Mean, Median and Mode}

In the year 2012, the average (mean) income of Americans is \$$40,563.00$, and the median income of Americans is \$$26,989.00$. Why are they different? How should we interpret these two numbers? This lesson will help you with these questions.

\subsection{Mean (average)}
In a class, a student's grade is determined by $5$ test scores. Each test has a total of $100$ points. Tom scored $70,80,85,20,90$. He did well except in the fourth test.

The teacher decides to use the mean (also called average) score to determine each student's grade. Here is how to calculate Tom's grade:
\[ \text{mean score}=\frac{70+80+85+20+85}{5}=\frac{340}{5}=68 \]
Since Tom's mean score is $68$, his final grade in this class was D.

You might think this grade is not fair, as Tom did well in 4 out of 5 tests, yet he still has to retake the class. What if the teacher uses median to decide Tom's grade?

\subsection{Median}
To calculate median of $70,80,85,20,90$, first we need to order the numbers from smallest to biggest:
\[ 20,70,80,85,90 \]
The number in the middle is the median, so Tom's median test score is $80$. If the teacher uses the median to decide a student's grade, Tom would earn B in this class.

What if there are $6$ tests in this class? Once we order the numbers, there would be two numbers in the middle. How would we determine the median, then?

Assume, in another class, Jerry scored $70,80,85,20,90,90$ in six tests. To find the median score, we first order the numbers:
\[ 20,70,80,85,90,90 \]
In the middle, there are two numbers: $80$ and $85$. The median of all six test scores is the mean of those two numbers in the middle:
\[ \text{median score}=\frac{80+85}{2}=82.5 \]
Jerry's median score in this class is $82.5$.

\subsection{Comparing Mean and Median}
In Tom's situation, should we use his mean score $68$ or his median score $80$ to determine his grade? If those are the only two options, I would use the median.

Toms' mean and median have a big difference because one of Tom's scores is $20$, way off the other scores. We call the value $20$ an \textit{outlier} in the group of numbers. Outliers can greatly affect the value of mean, but it doesn't affect the value of median. This is because, when we calculate median, we don't consider outliers---we only use the number (or two numbers) in the middle.

Now, think of the income situation. Why Americans' mean income is much higher than median income? Because there are outliers like the income of Bill Gates and Michael Jordan. Their income increased Americans' mean income by a large margin. However, when we calculate Americans' median income, Bill Gates' income is canceled by a poor person's income. This is why when we read newspapers, we more often see "median house value" or "median household income."

Let's look at a few more comparisons:
\begin{itemize}
\item If it's given the median daily income of $10$ people is \$$100.00$, we know $5$ people make more than (or exactly) \$$100.00$ every day, while the other $5$ people make less than (or exactly) \$$100.00$ every day. 
\item If it's given the mean daily income of $10$ people is \$$100.00$, we cannot tell how many people make more or less than \$$100.00$ every day. It could be every person in this group makes \$$100.00$ every day; it could be one person makes \$$1.00$ a day while another person makes \$$199.00$ a day.
\item If it's given the mean daily income of $10$ people is \$$100.00$, we know the total amount of daily income of these $10$ people is \$$100.00\cdot10=$\$$1,000.00$.
\item If it's given the median daily income of $10$ people is \$$100.00$, we cannot tell their total amount of daily income. It could be that one person makes \$$1.00$ a day while another makes \$$1,000,000$ a day (these two values canceled).
\end{itemize}

Here is the rule of thumb: When there are outliers, we should use median; when there are no outliers, we could use either mean or median (the values would be close anyway).

In this lesson's exercise, we calculate the mean and median of a few numbers. However, mean and median are only meaningful when there are many numbers.

\subsection{When Mean is Given}
We can do some calculations if the mean is given. Let's look at some examples.

\begin{myexample}
Five people chipped in an average of \$$75.00$ to purchase a computer. How much does the computer cost?
\end{myexample}
\begin{solution}
In this situation, it's like each person chipped in \$$75.00$ to purchase the computer. The total cost is:
\[ \$75.00 \cdot 5 = \$375.00 \]
\textbf{Conclusion:} The computer cost \$$375.00$.

Some people could have chipped in less, and some people could have chipped in more. We can ignore this fact because we only care about the total.
\end{solution}

\begin{myexample}
In a class, a student's final grade is decided by the average of $5$ test scores. Tom scored $80, 85, 90$ and $40$ in the first four tests. To earn C ($70\%$) in this class, what's the minimum score Tom must earn in the fifth test?
\end{myexample}
\begin{solution}
To earn an average of $70$ in five tests, Tom must score a total of $70\cdot5=350$ in five tests.

Tom has scored a total of $80+85+90+40=295$ in the first four tests.

He needs to score at least $350-295=65$ in the fifth test if he wants to earn an average of $70$ in these five tests.

\textbf{Conclusion:} Tom must earn at least $65$ in the fifth test in order to earn C in this class.

\end{solution}

\subsection{Mode}
Mode is the most frequent number is a group. For example, the mode of $1,2,2,3,21$ is $2$.

What's the mode of $1,2,2,3,3,21$? This data set has two modes: $2$ and $3$.

Mode is rarely used in real life.

