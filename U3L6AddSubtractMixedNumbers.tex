
\section{Add/Subtract Mixed Numbers}

In this lesson, we will learn how to add and subtract mixed numbers. We still need the skill of finding common denominators we learned earlier.

\subsection{Add/Subtract a Mixed Number and an Integer}

The following two examples should be easy to understand.

\begin{myexample}
\[ 
\begin{aligned}[t]
	&5\frac{2}{7}+3 = 8\frac{2}{7} \\
	&5\frac{2}{7}-3 = 2\frac{2}{7} \\
\end{aligned}
\]
\end{myexample}

The next example takes some thinking: When we subtract a mixed number like $5\frac{2}{7}$, it's equivalent to first subtracting $5$ whole pies, and then subtracting $\frac{2}{7}$ of a pie. Here are the first few steps of doing $7-5\frac{2}{7}$:

\[ 
\begin{aligned}[t]
	&\phantom{{}=} 7-5\frac{2}{7} \\
	&= 7-5-\frac{2}{7} \\
	&= 2-\frac{2}{7} \\
	&= ...
\end{aligned}
\]

Next, think about the situation: There are $2$ whole pies, and someone ate $\frac{2}{7}$ of one pie. There is still one whole pie left. The other pie, cut into $7$ pieces with $2$ pieces eaten, still has $5$ pieces left. So we have: $2-\frac{2}{7}=1\frac{5}{7}$. The full solution is:
\[ 
\begin{aligned}[t]
	&\phantom{{}=} 7-5\frac{2}{7} \\
	&= 7-5-\frac{2}{7} \\
	&= 2-\frac{2}{7} \\
	&= 1\frac{5}{7}
\end{aligned}
\]

Let's look at another example:
\begin{myexample}
\[ 
\begin{aligned}[t]
	&\phantom{{}=}10-4\frac{9}{20} \\
	&=10-4-\frac{9}{20} \\
	&=6-\frac{9}{20} \\
	&=5\frac{11}{20}
\end{aligned}
\]
\end{myexample}

It's more complicated when negative numbers are involved. See the next two examples.

\begin{myexample}
\[ 
\begin{aligned}[t]
	&\phantom{{}=}4-10\frac{9}{20} \\
	&=4-10-\frac{9}{20} \\
	&=-6-\frac{9}{20} \\
	&=-6\frac{9}{20}
\end{aligned}
\]
If you have trouble understanding the last step, think about $-1-2=-3$ (we need to add $1$ and $2$).
\end{myexample}

\begin{myexample}
\[ 
\begin{aligned}[t]
	&\phantom{{}=}-4-10\frac{9}{20} \\
	&=-4-10-\frac{9}{20} \\
	&=-14-\frac{9}{20} \\
	&=-14\frac{9}{20}
\end{aligned}
\]
\end{myexample}

\subsection{Add/Subtract Mixed Numbers with the Same Denominator}
The key is to break a mixed number into an integer and a fraction. Let's look at a few examples.

\begin{myexample}
\[ 
\begin{aligned}[t]
	&\phantom{{}=}2\frac{1}{6}+3\frac{1}{6} \\
	&= 2+\frac{1}{6}+3+\frac{1}{6} \\
	&= 2+3+\frac{1}{6}+\frac{1}{6} \\
	&= 5+\frac{1+1}{6} \\
	&= 5+\frac{2}{6} \\
	&= 5\frac{1}{3}
\end{aligned}
\]
\end{myexample}

The next example is more complicated.

\begin{myexample}
\[ 
\begin{aligned}[t]
	&\phantom{{}=}2\frac{5}{6}+3\frac{5}{6} \\
	&= 2+\frac{5}{6}+3+\frac{5}{6} \\
	&= 2+3+\frac{5}{6}+\frac{5}{6} \\
	&= 5+\frac{5+5}{6} \\
	&= 5+\frac{10}{6} \\
	&= 5+\frac{5}{3} \\
	&= 5+1\frac{2}{3} \\
	&= 6\frac{2}{3}
\end{aligned}
\]
In this example, we changed $\frac{5}{3}$ to $1\frac{2}{3}$.
\end{myexample}

The next example shows how to do mixed number subtraction. Again, the key is to break the mixed number into an integer and a fraction.

\begin{myexample}
\[ 
\begin{aligned}[t]
	&\phantom{{}=}7\frac{5}{6}-4\frac{1}{6} \\
	&= 7+\frac{5}{6}-4-\frac{1}{6} \\
	&= 7-4+\frac{5}{6}-\frac{1}{6} \\
	&= 3+\frac{5-1}{6} \\
	&= 3+\frac{4}{6} \\
	&= 3\frac{2}{3}
\end{aligned}
\]
\end{myexample}

The next example is more challenging. We need to review how to do subtraction like $31-17$. Once we line up those two numbers, we have:
\[
\begin{aligned}[t]
	&\phantom{-}31 \\
	&\underline{-17}
\end{aligned}
\]

Since we cannot do $1-7$ in the ones place, we use the concept of "borrowing" by taking $10$ from $30$, and put the $10$ to the $1$ in the ones place. Now, the number $31$ is broken into $20$ and $11$.

Now we can do $20-10$ in the tens place, and $11-7$ in the ones place, and the final answer is $14$.

We will use the same concept to do the following mixed number subtraction problem:

\begin{myexample}
\[ 
\begin{aligned}[t]
	&\phantom{{}=}7\frac{1}{6}-4\frac{5}{6} \\
	&= 7+\frac{1}{6}-4-\frac{5}{6} \\
	&= 7-4+\frac{1}{6}-\frac{5}{6} \\
	&= 3+\frac{1}{6}-\frac{5}{6} \\
	&= 2+1+\frac{1}{6}-\frac{5}{6} & \text{borrowing 1 from 3}\\
	&= 2+\frac{6}{6}+\frac{1}{6}-\frac{5}{6} &\text{change }1\text{ to } \frac{6}{6}\\
	&= 2+\frac{7}{6}-\frac{5}{6} \\
	&= 2+\frac{2}{6} \\
	&= 2\frac{1}{3} &\text{reduce fraction} \\
\end{aligned}
\]
In this example, since we cannot subtract $\frac{5}{6}$ from $\frac{1}{6}$, we "borrowed" $1$ from the integer $3$, changed $1$ to $\frac{6}{6}$, and then changed $\frac{1}{6}$ to $\frac{6}{6}+\frac{1}{6}=\frac{7}{6}$. Now we can subtract $\frac{5}{6}$ from $\frac{7}{6}$. This is the same strategy we used when we do $31-17=14$.
\end{myexample}

Things become more complicated when negative numbers are involved. See the next few examples.

\begin{myexample}
\[ 
\begin{aligned}[t]
	&\phantom{{}=}-7\frac{1}{6}-4\frac{5}{6} \\
	&= -7-\frac{1}{6}-4-\frac{5}{6} \\
	&= -7-4-\frac{1}{6}-\frac{5}{6} \\
	&= -11+(-\frac{1}{6})+(-\frac{5}{6}) \\
	&= -11+\frac{(-1)+(-5)}{6} \\
	&= -11+\frac{-6}{6} \\
	&= -11+(-1) \\
	&= -12
\end{aligned}
\]
To make it clear, we changed $-\frac{1}{6}-\frac{5}{6}$ to $(-\frac{1}{6})+(-\frac{5}{6})$. This is like changing $-1-2$ to $(-1)+(-2)$.
\end{myexample}

\begin{myexample}
\[ 
\begin{aligned}[t]
	&\phantom{{}=}-7\frac{5}{6}+4\frac{1}{6} \\
	&= -7-\frac{5}{6}+4+\frac{1}{6} \\
	&= -7+4-\frac{5}{6}+\frac{1}{6} \\
	&= -3+\frac{-5}{6}+\frac{1}{6} \\
	&= -3+\frac{-5+1}{6} \\
	&= -3+\frac{-4}{6} \\
	&= -3+\frac{-2}{3} \\
	&= -3\frac{2}{3}
\end{aligned}
\]
The last step takes some thinking. Think of $(-1)+(-2)=-3$. When we add two negative numbers, we actually add up the absolute value of those two numbers. This is why when we do $-3+\frac{-2}{3}$, we need to do $3+\frac{2}{3}=3\frac{2}{3}$, and then make it negative.
\end{myexample}

\begin{myexample}
\[ 
\begin{aligned}[t]
	&\phantom{{}=}1\frac{5}{6}-4 \\
	&= 1+\frac{5}{6}-4 \\
	&= 1-4+\frac{5}{6} \\
	&= -3+\frac{5}{6} \\
	&= -2 -1 + \frac{5}{6} \\
	&= -2 -\frac{6}{6} + \frac{5}{6} \\
	&= -2 + \frac{-6+5}{6} \\
	&= -2 + \frac{-1}{6} \\
	&= -2\frac{1}{6}
\end{aligned}
\]
\end{myexample}

Adding/subtracting mixed numbers with different denominators is more complicated, but the strategies are the same.

\subsection{Summary}
When we do mixed number addition/subtraction, we usually break up the mixed number into its integer part and fraction part. For example:
\[ 2\frac{3}{5} = 2+\frac{3}{5} \]
\[ -2\frac{3}{5} = -2-\frac{3}{5} \]

A lot of practice is needed for mixed number addition/subtraction. Make sure you understand each step as you go.

