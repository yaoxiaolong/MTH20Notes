
\section{Decimals and Fractions}

For $\frac{1}{2}$ of a dollar, we could also say \$$0.50$. In other words, $\frac{1}{2}=0.5$. In this lesson, we will learn how to convert between decimals and fractions.

\subsection{Overview}
Let's review the definition of fractions:
\begin{itemize}
\item $\frac{1}{2}$ means we divide the whole into $2$ pieces evenly, and then take $1$ piece.
\item $\frac{1}{5}$ means we divide the whole into $5$ pieces evenly, and then take $1$ piece.
\item $\frac{3}{4}$ means we divide the whole into $4$ pieces evenly, and then take $3$ pieces.
\end{itemize}

We can understand fractions in terms of money, if we treat the "whole" as a dollar:
\begin{itemize}
\item $\frac{1}{2}$ means we divide \$$1$ into $2$ pieces evenly (each piece is $50$ cents), and then take $1$ piece, which is $50$ cents.
\item $\frac{1}{5}$ means we divide \$$1$ into $5$ pieces evenly (each piece is $20$ cents), and then take $1$ piece, which is $20$ cents.
\item $\frac{3}{4}$ means we divide \$$1$ into $4$ pieces evenly (each piece is $25$ cents), and then take $3$ pieces, which is $75$ cents.
\end{itemize}

When we think about money, the following equations should make sense (think through each):
\[
\begin{aligned}[t]
	\frac{1}{2} &= 0.5 &&50 \text{ cents}\\
	\frac{1}{5} &= 0.2 &&20 \text{ cents}\\
	\frac{3}{4} &= 0.75 &&75 \text{ cents}\\
	\frac{1}{50} &= 0.02 &&2 \text{ cents}\\
	\frac{1}{20} &= 0.05 &&5 \text{ cents}\\
	\frac{99}{100} &= 0.99 &&99 \text{ cents}
\end{aligned}
\]

\subsection{Convert Fraction to Decimal}
Converting from fraction to decimal is easy, as the fraction line simply means division. Let's look at a few examples:
\[
\begin{aligned}[t]
	\frac{1}{2} &= 1\div2 = 0.5 \\
	\frac{1}{5} &= 1\div5 = 0.2 \\
	\frac{3}{4} &= 3\div4 = 0.75 \\
\end{aligned}
\]

Unless the numbers are simple enough that you can do the division in your head, feel free to use a calculator. Rarely would an instructor ask you to convert $\frac{3}{16}$ to a decimal by doing long division on scratch paper.

When the decimal is not a terminating decimal, usually we round the decimal to either the tenths place or the hundredths place.

\begin{myexample}
Convert $\frac{2}{3}$ to a decimal. Round to the tenth place.
\end{myexample}
\begin{solution}
To change $\frac{2}{3}$ to a decimal, we use division:
\[ \frac{2}{3}=2\div3=0.666...\approx 0.7 \]
The digit after the tenths place is $6$, so we round up and get $0.7$.
\end{solution}

\begin{myexample}
Convert $\frac{2}{3}$ to a decimal. Round to the hundredth place.
\end{myexample}
\begin{solution}
To change $\frac{2}{3}$ to a decimal, we use division:
\[ \frac{2}{3}=2\div3=0.666...\approx 0.67 \]
The digit after the hundredth place is $6$, so we round up and get $0.67$.
\end{solution}

\begin{myexample}
Convert $\frac{7}{12}$ to a decimal. Round to the hundredth place.
\end{myexample}
\begin{solution}
To change $\frac{7}{12}$ to a decimal, we use division:
\[ \frac{7}{12}=7\div12=0.58333...\approx 0.58 \]
The digit after the hundredth place is $3$, so we do not round up and get $0.58$.
\end{solution}

\subsection{Convert Decimal to Fraction}
To convert a decimal to a fraction, we need to correctly read the decimal, and the conversion will be easy. Here are a few examples:

\begin{itemize}
\item We read $0.5$ as "five tenth", which is $\frac{5}{10}$. We can further reduce the fraction to $\frac{1}{2}$.
\item We read $0.05$ as "five hundredth", which is $\frac{5}{100}$. We can further reduce the fraction to $\frac{1}{20}$.
\item We read $0.125$ as "five thousandth", which is $\frac{125}{1000}$. We can further reduce the fraction:
\[ 0.125=\frac{125}{1000}=\frac{125\div5}{1000\div5}=\frac{25}{200}=\frac{25\div5}{200\div5}=\frac{5}{40}=\frac{5\div5}{40\div5}=\frac{1}{8} \]
\end{itemize}

Earlier, we learned that trailing zeroes don't change the value of a decimal. For example, $0.2=0.20$. Here is a chance to further understand this:
\begin{itemize}
\item We read $0.2$ as "two tenth", which is $\frac{2}{10}$. We can further reduce the fraction:
\[ 0.2=\frac{2}{10}=\frac{2\div2}{10\div2}=\frac{1}{5} \]
\item We read $0.20$ as "twenty hundredth", which is $\frac{20}{100}$. We can further reduce the fraction:
\[ 0.20=\frac{20}{100}=\frac{20\div20}{100\div20}=\frac{1}{5} \]
\end{itemize}

Memorizing the following commonly-used conversions will be critical to building your number sense. Most of the conversions below can be understood by the money model. For example, both $\frac{1}{2}$ and $0.5$ represent $50$ cents.

\[
\begin{aligned}
	\frac{1}{2}&=0.5 \\
	\frac{1}{3}&=0.\overline{3} &\phantom{aaa} \frac{2}{3}&=0.\overline{6} \\
	\frac{1}{4}&=0.25 &\phantom{aaa} \frac{3}{4}&=0.75 \\
	\frac{1}{5}&=0.2 &\phantom{aaa} \frac{2}{5}&=0.4 &\phantom{aaa} \frac{3}{5}&=0.6 &\phantom{aaa} \frac{4}{5}&=0.8 \\
	\frac{1}{6}&=0.1\overline{6} &\phantom{aaa} \frac{5}{6}&=0.8\overline{3} \\
	\frac{1}{8}&=0.125 &\phantom{aaa} \frac{3}{8}&=0.375 &\phantom{aaa} \frac{5}{8}&=0.625 &\phantom{aaa} \frac{7}{8}&=0.875 \\
	\frac{1}{9}&=0.\overline{1} &\phantom{aaa} \frac{2}{9}&=0.\overline{2} &\phantom{aaa} &... &\phantom{aaa} \frac{8}{9}&=0.\overline{8} \\
	\frac{1}{10}&=0.1 &\phantom{aaa} \frac{3}{10}&=0.3 &\phantom{aaa} \frac{7}{10}&=0.7 &\phantom{aaa} \frac{9}{10}&=0.9 \\
	\frac{1}{20}&=0.05 &\phantom{aaa} \frac{3}{20}&=0.15 &\phantom{aaa} &... &\phantom{aaa} \frac{19}{20}&=0.95 \\
	\frac{1}{25}&=0.04 &\phantom{aaa} \frac{2}{25}&=0.08 &\phantom{aaa} &... &\phantom{aaa} \frac{24}{25}&=0.96 \\
\end{aligned}
\]

\subsection{Summary}
\begin{itemize}
\item To change a fraction to decimal, we simply do a division. For example, $\frac{1}{2}=1\div2=0.5$.
\item To change a decimal to fraction, we read the decimal, write the fraction, and then reduce if possible. For example, $0.5$ is read as "five tenth." We write it as $\frac{5}{10}$, and then reduce it to $\frac{1}{2}$.
\end{itemize}

