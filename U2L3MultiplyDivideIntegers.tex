
\section{Multiply and Divide Integers}
\thispagestyle{fancy}
In this lesson, we will learn how to multiply and divide positive and negative integers.

\subsection{Multiply Integers}
Assume you are walking on a number line. Remember a number line has a positive direction and a negative direction. Assume you are facing the positive direction.

If you walk backward at a speed of $2$ feet per step, you would walk in the negative direction. We can say you walk $-2$ feet per step.

In $3$ steps, you would walk $6$ feet in the negative direction. We could do

\[ (-2)+(-2)+(-2)=-6 \]

Easier, we can use multiplication:

\[ (-2)\cdot3=-6 \]

We can see this in the figure:

		\begin{tightcenter}
			\begin{tikzpicture}
				\begin{axis}[
						xmin=-8,xmax=8,
						ymin=-1,ymax=2,
						axis y line=none,
						height =1cm,
						grid=none,
						xtick={-7,-6,...,7},
						xlabel={}
					]
					\addplot[<-,line width=3pt,red,domain=-2:0]{0.5} node[pos=0.5,anchor=south]{$-2$};
					\addplot[<-,line width=3pt,red,domain=-4:-2]{0.5} node[pos=0.5,anchor=south]{$-2$};
					\addplot[<-,line width=3pt,red,domain=-6:-4]{0.5} node[pos=0.5,anchor=south]{$-2$};	
					\addplot[soldot]coordinates{ (0,0) };			
					\addplot[soldot]coordinates{ (-2,0) };
					\addplot[soldot]coordinates{ (-4,0) };
					\addplot[soldot]coordinates{ (-6,0) };
				\end{axis}
			\end{tikzpicture}
			\captionof{figure}{$(-2)\cdot3=-6$}
		\end{tightcenter}

We can see:
\[ \text{(negative)}\cdot\text{(positive)}=\text{negative} \]

The next rule is:
\[ \text{(negative)}\cdot\text{(negative)}=\text{positive} \]

There are two ways to understand this.

An easy way is to think about this sentence: I do NOT NOT like football. Since double-negation implies affirmation, this sentence actually means "I do like football." This is why $ \text{(negative)}\cdot\text{(negative)}=\text{positive} $.

Another way is to use the number line. In $(-2)\cdot(-3)$, the number $-2$ implies a person walks \textit{backward} $2$ feet per step; the number $-3$ implies this person walks $3$ steps while \textit{facing the negative direction}. In this situation, since the person faces the negative direction while walking backward, the person is actually moving \textit{in the positive direction}! See the following figure:

		\begin{tightcenter}
			\begin{tikzpicture}
				\begin{axis}[
						xmin=-8,xmax=8,
						ymin=-1,ymax=2,
						axis y line=none,
						height =1cm,
						grid=none,
						xtick={-7,-6,...,7},
						xlabel={}
					]
					\addplot[->,line width=3pt,red,domain=0:2]{0.5} node[pos=0.5,anchor=south]{$2$};
					\addplot[->,line width=3pt,red,domain=2:4]{0.5} node[pos=0.5,anchor=south]{$2$};
					\addplot[->,line width=3pt,red,domain=4:6]{0.5} node[pos=0.5,anchor=south]{$2$};	
					\addplot[soldot]coordinates{ (0,0) };			
					\addplot[soldot]coordinates{ (2,0) };
					\addplot[soldot]coordinates{ (4,0) };
					\addplot[soldot]coordinates{ (6,0) };
				\end{axis}
			\end{tikzpicture}
			\captionof{figure}{$(-2)\cdot(-3)=6$}
		\end{tightcenter}

Let's summarize multiplication rules:
\[
\begin{aligned}[t]
	&\text{(positive)}\cdot\text{(positive)}=\text{positive}&&\text{example}: 2\cdot3=6\\
	&\text{(negative)}\cdot\text{(positive)}=\text{negative}&&\text{example}: (-2)\cdot3=-6\\
	&\text{(positive)}\cdot\text{(negative)}=\text{negative}&&\text{example}: 2\cdot(-3)=-6\\
	&\text{(negative)}\cdot\text{(negative)}=\text{positive}&&\text{example}: (-2)\cdot(-3)=6\\
\end{aligned}
\]

\begin{myexample}
Calculate $(-1)(-2)(-3)$
\end{myexample}
\begin{solution}
We learned that, in multiplication, $\text{(negative)}\cdot\text{(negative)}=\text{positive}$. In other words, each pair of negative signs cancel each other.

In this problem, there are three negative signs. After one pair cancel each other, there is still one left, making the product negative. We have:
	\[ (-1)(-2)(-3)=-1\cdot2\cdot3 \]
After finding the product is negative, the problem is easier. Since $1\cdot2\cdot3=6$, we have:
	\[ (-1)(-2)(-3)=-1\cdot2\cdot3 = -6 \]
\end{solution}

\begin{myexample}
Calculate $(-1)(-2)(-3)(-4)$
\end{myexample}
\begin{solution}
In this problem, there are four negative signs. After two pairs of negative signs cancel each other, no negative signs are left, making the product positive. We have:
	\[ (-1)(-2)(-3)(-4)=1\cdot2\cdot3\cdot4 \]
After finding the product is positive, the problem is easier:
	\[ (-1)(-2)(-3)(-4)=1\cdot2\cdot3\cdot4 = 24 \]
\end{solution}

\subsection{Divide Integers}
Division rules are the same as multiplication rules:
\[
\begin{aligned}[t]
	&\frac{\text{(positive)}}{\text{(positive)}}=\text{positive}&&\text{example}: \frac{6}{2}=3\\
	&\frac{\text{(negative)}}{\text{(positive)}}=\text{negative}&&\text{example}: \frac{-6}{2}=-3\\
	&\frac{\text{(positive)}}{\text{(negative)}}=\text{negative}&&\text{example}: \frac{6}{-2}=-3\\
	&\frac{\text{(negative)}}{\text{(negative)}}=\text{positive}&&\text{example}: \frac{-6}{-2}=3\\
\end{aligned}
\]

\subsection{Exponent of Negative Numbers}
Let's find a pattern:
\[
\begin{aligned}[t]
	&(-2)^{1}=-2 \\
	&(-2)^{2}=(-2)(-2)=4 \\
	&(-2)^{3}=(-2)(-2)(-2)=-8 \\
	&(-2)^{4}=(-2)(-2)(-2)(-2)=16 \\
	&(-2)^{5}=(-2)(-2)(-2)(-2)(-2)=-32 \\
	&...
\end{aligned}
\]
If we raise a negative number to an odd exponent, the product is negative; if we raise a negative number to an even exponent, the product is positive. This is because each pair of negative signs cancel each other. So we have:
\[
\begin{aligned}[t]
	&(-1)^{100}=1 \\
	&(-1)^{99}=-1 \\
\end{aligned}
\]
We need to learn an important difference:
\[
\begin{aligned}[t]
	&-3^{2}=-9 \\
	&(-3)^{2}=9 \\
\end{aligned}
\]
To explain the difference, we need to find a pattern first:
\[
\begin{aligned}[t]
	&-2=-1\cdot2 \\
	&-3=-1\cdot3 \\
	&-4=-1\cdot4 \\
	&-5=-1\cdot5 \\
	&...
\end{aligned}
\]
We can see one way to understand the negative sign is "$(-1)$ times", so we can re-write $-3^{2}$ as $(-1)\cdot3^{2}$.

Next, by the order of operations, PEMDAS, exponent overrides multiplication, so we need to handle exponent first: $(-1)\cdot3^{2}=(-1)\cdot9$.

The full solution is:
\[
\begin{aligned}[t]
	&\phantom{{}=}-3^{2} \\
	&=(-1)\cdot3^{2} \\
	&=(-1)\cdot9 \\
	&=-9
\end{aligned}
\]

Next, let's look at $(-3)^{2}$. In this problem, by PEMDAS, parentheses override exponent, so we cannot do $3^{2}$ first as we did in $-3^{2}$. Instead, we do:
\[
\begin{aligned}[t]
	&\phantom{{}=}(-3)^{2} \\
	&=(-3)\cdot(-3) \\
	&=9
\end{aligned}
\]

In summary, we have:
\[
\begin{aligned}[t]
	&-3^{2}=-9 \\
	&(-3)^{2}=9 \\
\end{aligned}
\]
This is decided by the order of operations.

However, don't try to memorize rules. Instead, understand the reasoning. Rules change when situations change. See the next example.

\begin{myexample}
Evaluate the following:
\[
\begin{aligned}[t]
	&-2^{3} \\
	&(-2)^{3} \\
\end{aligned}
\]
\end{myexample}
\begin{solution}
\[
\begin{aligned}[t]
	&-2^{3}=-2\cdot2\cdot2=-8 \\
	&(-2)^{3}=(-2)(-2)(-2)=-2\cdot2\cdot2=-8 \\
\end{aligned}
\]
\end{solution}
When a negative number is raised to an odd exponent, the result is always negative, with or without parentheses. Again, I don't want you to memorize one more rule. Instead, try to understand the math behind it.


