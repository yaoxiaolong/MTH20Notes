

\chapter{Percent}
\section{Introduction to Percent}
\thispagestyle{fancy}

We use fractions to represent part of a number. Because we are using a decimal number system, it's natural to divide the whole into $100$ pieces. Instead of using fractions like $\frac{7}{100}$, we use percent like $7\%$.

\subsection{Definition of Percent}
The $\%$ symbol means $\frac{}{100}$. For example:
\begin{itemize}
\item $50\%=\frac{50}{100}$
\item $5\%=\frac{5}{100}$
\item $120\%=\frac{120}{100}$
\end{itemize}

It helps to understand percent by thinking of money. For example, $50\%$ means $50$ cents; $5\%$ means $5$ cent; and $120\%$ means $120$ cents.

\subsection{Percent and Decimal}
Let's look at a few examples:
\begin{itemize}
\item $50\%$ = $0.5$ ($50$ cents)
\item $5\%$ = $0.05$ ($5$ cents)
\item $120\%$ = $1.2$ ($120$ cents)
\end{itemize}

The pattern is: To change a number from percent to decimal, move the decimal point to the right twice. Instead of memorizing this rule, write down $50\%$ = $0.5$ on scratch paper and you can easily see the rule.

Now it's easy to understand the following more complicated conversions:
\begin{itemize}
\item $100\%$ = $1$ ($100$ cents)
\item $200\%$ = $2$ ($200$ cents)
\item $0.1\%$ = $0.001$ (one tenth of a cent)
\item $22.5\%$ = $0.225$ (twenty-two and a half cents)
\end{itemize}

\subsection{Percent and Fraction}
To change percent to fraction, we first change percent to decimal, and then change decimal to fraction. Don't forget to reduce fractions!
\begin{itemize}
\item $50\%$ = $0.5$ (read as "five tenths") = $\frac{5}{10}$ = $\frac{1}{2}$
\item $5\%$ = $0.05$ (read as "five hundredth") = $\frac{5}{100}$ = $\frac{1}{20}$
\item $12.5\%$ = $0.125$ (read as "one hundred twenty-five thousandth") = $\frac{125}{1000}$ = $\frac{125\div125}{1000\div125}$ = $\frac{1}{8}$
\item $150\%$ = $1.5$ (read as "one and five tenth") = $1\frac{5}{10}$ = $1\frac{1}{2}$
\end{itemize}

To change fraction to percent, we first change fraction to decimal, and then change decimal to percent. Let's look at a few examples:
\begin{itemize}
\item $\frac{1}{2} = 1\div2 = 0.5 = 50\%$
\item $\frac{7}{5} = 7\div5 = 1.4 = 140\%$
\end{itemize}

Sometimes we need to round the percent. In the following examples, we will round the percent to two decimal places.
\begin{itemize}
\item $\frac{2}{3} = 2\div3 = 0.66666... \approx 0.6667 = 66.67\%$
\item $\frac{8}{7} = 8\div7 = 1.142857...\approx 1.1429 = 114.29\%$
\end{itemize}

\subsection{Percent, Decimal and Fraction}
The following conversions will be used very often. They are critical to building your number sense.

\begin{center}
\begin{tabular}{ | c | c | c | }
	\hline
    \textbf{Fraction} & \textbf{Decimal} & \textbf{Percent} \\ \hline
  $\frac{1}{2}$ & $0.5$ & $50\%$ \\ \hline
  $\frac{1}{3}$ & $0.\overline{3}$ & $\approx33.33\%$ \\ \hline
  $\frac{2}{3}$ & $0.\overline{6}$ & $\approx66.67\%$ \\ \hline
  $\frac{1}{4}$ & $0.25$ & $25\%$ \\ \hline
  $\frac{3}{4}$ & $0.75$ & $75\%$ \\ \hline
  $\frac{1}{5}$ & $0.2$ & $20\%$ \\ \hline
  $\frac{2}{5}$ & $0.4$ & $40\%$ \\ \hline
  $\frac{3}{5}$ & $0.6$ & $60\%$ \\ \hline
  $\frac{4}{5}$ & $0.8$ & $80\%$ \\ \hline
  $\frac{1}{8}$ & $0.125$ & $12.5\%$ \\ \hline
  $\frac{3}{8}$ & $0.375$ & $37.5\%$ \\ \hline
  $\frac{5}{8}$ & $0.625$ & $62.5\%$ \\ \hline
  $\frac{7}{8}$ & $0.875$ & $87.5\%$ \\ \hline
  $\frac{1}{10}$ & $0.1$ & $10\%$ \\ \hline
  $\frac{3}{10}$ & $0.3$ & $30\%$ \\ \hline
  $\frac{7}{10}$ & $0.7$ & $70\%$ \\ \hline
  $\frac{9}{10}$ & $0.9$ & $90\%$ \\ \hline
\end{tabular}
\end{center}

\subsection{Meaning of More Than $100\%$}
Assume School A has $1,000$ students. If School B has $500$ students, then School B has $50\%$ of School A's students. This should be easy to understand.

What if School B has $1,500$ students? This is one and a half times of School A's students. We say School B has $150\%$ of School A's students.

We use the same rules when we convert $150\%$ to decimal and fraction, except the fraction should be a mixed number:
\[ 150\% = 1.5 = 1 \frac{1}{2} \]

After understanding this section, you should have the number sense that $200\%$ means "twice", and $300\%$ means "three times. A percent bigger than $100\%$ means "more than the whole."

