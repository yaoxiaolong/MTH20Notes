
\section{Prime Number}

Prime numbers are the backbone of all numbers. In this lesson, we will learn prime numbers and how to prime factor a number.

\subsection{Factor and Prime}
Since $\frac{8}{2}=4$, we say $2$ is a factor of $8$, because $2$ goes into $8$ four times with no remainder.

Let's look at factors of the first few numbers, starting with $2$:
\begin{itemize}
\item $2$ has two factors: $1$ and $2$.
\item $3$ has two factors: $1$ and $3$.
\item $4$ has three factors: $1,2$ and $4$.
\item $5$ has two factors: $1$ and $5$.
\item $6$ has four factors: $1,2,3$ and $6$.
...
\end{itemize}
We can see a number (except $1$) has at least two factors: $1$ and the number itself.

If a number has only two factors, $1$ and the number itself, this number is a \textit{prime number}. For example, $2$, $3$ and $5$ are prime numbers.

If a number has more than two factors, this number is a \textit{composite number}. For example, $4$ and $6$ are composite numbers.

If a number is an even number bigger than $2$, it has at least $3$ factors: $1$, $2$ and itself. All even numbers, except $2$, are composite numbers.

It's useful to memorize the first few prime numbers, which will be used frequently later: $2,3,5,7,11,13,...$

Note that $1$ is neither a prime number nor a composite number.

\begin{myexample}
Is $245$ prime or composite?
\end{myexample}
\begin{solution}
When we judge whether a number is prime or composite, we look at the first few prime numbers, $2,3,5,7,11,13,...$, and see whether any of them goes into the number evenly.

\begin{itemize}
\item $2$ doesn't go into $245$ evenly, because $245$ is odd.
\item $3$ doesn't go into $245$ evenly, because $3$ doesn't go evenly into $2+4+5=11$.
\item $5$ goes into $245$ evenly, because the last digit of $245$ is $5$.
\end{itemize}
We can tell $245$ is composite because $5$ goes into $245$ evenly, implying $245$ has at least $3$ factors: $1$, $5$ and $245$.
\end{solution}

\begin{myexample}
Is $199$ prime or composite?
\end{myexample}
\begin{solution}
When we judge whether a number is prime or composite, we look at the first few prime numbers, $2,3,5,7,11,13,...$, and see whether any of them goes into the number evenly.

\begin{itemize}
\item $2$ doesn't go into $199$ evenly, because $199$ is odd.
\item $3$ doesn't go into $199$ evenly, because $3$ doesn't go evenly into $1+9+9=19$.
\item $5$ doesn't go into $199$ evenly, because the last digit of $199$ is not $0$ or $5$.
\item To judge whether $7$ goes into $199$ evenly, it's reasonable to use a calculator. We can see $\frac{199}{7}=28.42\hdots$, so $7$ doesn't go into $199$ evenly.
\item With the help of calculator, we can tell neither $11$ nor $13$ goes into $199$ evenly. 
\end{itemize}

Theoretically, we must test all prime numbers smaller than $199$: $17,19,23,...$. In reality, no reasonable instructor would ask you to spend so much time to judge whether a number if prime or composite. After you test $2,3,5,7$ and $11$, and if none of them goes into the number evenly, it's reasonable to say the number is prime.

For this example, $199$ is indeed a prime number.
\end{solution}

\subsection{Prime Factoring}
Prime numbers are backbones of numbers. Each composite number can be written as the product of prime numbers. For example, $15=3\cdot5$ and $12=2\cdot2\cdot3$. The last equation can be re-written in a simpler way as $12=2^{2}\cdot3$. We will learn how to prime factor numbers.

\begin{myexample}
Prime factor $12$.
\end{myexample}
\begin{solution}
First, write down the first few prime numbers: $2,3,5,7,11$. Rarely would an instructor expect you to use bigger prime numbers in prime factoring. We will prime factor $12$ as an example.

To prime factor the number $12$, we look at the first prime number, $2$, and see whether $2$ goes into $12$. It does! We have $12=2\cdot6$.

In $2\cdot6$, the number $2$ is prime, and we cannot further break it down; the number $6$ is not prime, and can be further factored.

Again, we go back to the list of prime numbers, and see whether the first prime number $2$ goes into $6$. It does! Now we have $12=2\cdot6=2\cdot2\cdot3$. We can stop here because all numbers in the product $2\cdot2\cdot3$ are prime numbers.

We can re-write the product with exponent: $12=2^{2}\cdot3$.

It's easier to use a tree to prime factor numbers:
\[
\begin{forest}
   [12
      [2]
      [6
         [2]
         [3]
      ]
   ]
\end{forest}
\]
\end{solution}

Note that we don't use the number $1$ in prime factoring. Otherwise, we could keep going with no end like $12=2^{2}\cdot3\cdot1\cdot1\cdot...$.

\begin{myexample}
Prime factor 250.
\end{myexample}
\begin{solution}
We will use a factor tree to solve this problem.
\[
\begin{forest}
   [250
      [2]
      [125
         [5]
         [25
            [5]
            [5]
         ]
      ]
   ]
\end{forest}
\]
The prime factor of $250$ is $250=2\cdot5^{3}$.
\end{solution}

We can see a big number like $250$ can be broken down to the product of very small prime numbers.

