
\section{Multiply/Divide Decimals}

In this lesson, we will learn how to multiply/divide decimals. Again, please refrain from using calculator in this section (actually in this whole course).

\subsection{Multiply/Divide by Power of 10}
Recall that for the number $12$, we could write it as $12.0$, $12.00$ or $12.000$, as the value doesn't change.

Let's observe a pattern. In the following numbers, I added trailing zeroes and decimal point to the end of integers in order to find the pattern:
\[
\begin{aligned}[t]
	12.0\cdot10 &= 120. \\
	12.00\cdot100 &= 1200. \\
	12.000\cdot1000 &= 12000. \\
\end{aligned}
\]

Here is the pattern: \textbf{When we multiply a number by a power of $10$, like $10,100,1000$..., we move the decimal point to the right as many times as the number of zeroes.}

A similar pattern exists for division:
\[
\begin{aligned}[t]
	12000.\div10 &= 1200. \\
	12000.\div100 &= 120. \\
	12000.\div1000 &= 12. \\
\end{aligned}
\]

Here is the pattern: \textbf{When we divide a number by a power of $10$, like $10,100,1000$..., we move the decimal point to the left as many times as the number of zeroes.}

Don't try to memorize the patterns without understanding. If you forget in which direction the decimal point will move, on scratch paper, do $12.\cdot10=120.$ and $120.\div10=12.$, and then you can quickly recall the patterns.

The same patterns work for decimals. Here are a few examples:
\[
\begin{aligned}[t]
	123.456\cdot10 &= 1234.56 &&\text{Decimal point moves to the right once.}\\
	123.456\cdot100 &= 12345.6 &&\text{Decimal point moves to the right twice.}\\
	123.456\div10 &= 12.3456 &&\text{Decimal point moves to the left once.}\\
	123.456\div100 &= 1.23456 &&\text{Decimal point moves to the left twice.}
\end{aligned}
\]

Sometimes we need to fill in zeroes:
\[
\begin{aligned}[t]
	1.2\cdot100 &= 120 &&\text{Decimal point moves to the right twice.}\\
	1.2\cdot1000 &= 1200 &&\text{Decimal point moves to the right three times.}\\
	1.2\div10 &= 0.12 &&\text{Decimal point moves to the left once.}\\
	1.2\div100 &= 0.012 &&\text{Decimal point moves to the left twice.}\\
	1.2\div1000 &= 0.0012 &&\text{Decimal point moves to the left three times.}\\
\end{aligned}
\]

\subsection{Moving Decimal Point in Multiplication}
It's given that $12\cdot12=144$. What's the product of $12\cdot1.2$?

From the pattern we learned earlier, we know $1.2=12\div10$. So we have:
\[ 
\begin{aligned}[t]
	&\phantom{{}=}12\cdot1.2 \\
	&=12\cdot12\div10 \\
	&=144\div10 \\
	&=14.4
\end{aligned}
\]

Without giving details, with the same method, we can calculate:
\[ 
\begin{aligned}[t]
	12\cdot0.12 &= 1.44 \\
	12\cdot0.012 &= 0.144 \\
	1.2\cdot1.2 &= 1.44 \\
	0.12\cdot12 &= 1.44 \\
	0.12\cdot1.2 &= 0.144
\end{aligned}
\]

Instead of writing a long sentence to summarize the pattern, I will simply use an example. It's given $12\cdot12=144$. When we calculate $0.12\cdot1.2$, observe that the decimal point moved to the left for a total of three times, compared with $12\cdot12$. So, we need to move the decimal point of $144$ to the left three times, and we have $0.12\cdot1.2=0.144$.

The pattern can be applied when we move the decimal point to the left or right. Let's look at a few examples.

\begin{myexample}
Given $123\cdot45=5535$, calculate $1.23\cdot45000$.
\end{myexample}
\begin{solution}
\begin{itemize}
\item From $123$ to $1.23$, the decimal point moved to the left twice.
\item From $45$ to $45000$, the decimal point moved to the right three times.
\item Altogether, the decimal point moved to the right once.
\item We move $5535$'s decimal point to the right once, and we have: $1.23\cdot45000=55350$.
\end{itemize}
\end{solution}

\begin{myexample}
Given $123\cdot45=5535$, calculate $1.23\cdot4500$.
\end{myexample}
\begin{solution}
\begin{itemize}
\item From $123$ to $1.23$, the decimal point moved to the left twice.
\item From $45$ to $4500$, the decimal point moved to the right twice.
\item Altogether, the decimal point didn't move.
\item The decimal point of the product $5535$ didn't move, and we have: $1.23\cdot4500=5535$.
\end{itemize}
\end{solution}

\subsection{Moving Decimal Point in Division}
There are similar rules of moving the decimal point when we do divisions. Let me explain the rules with fractions.

Recall that for a fraction, if we multiply or divide the same number in both the numerator and denominator, the fraction's value doesn't change. For example:
\[ \frac{1}{2}=\frac{1\cdot2}{2\cdot2}=\frac{2}{4} \]
\[ \frac{2}{4}=\frac{2\div2}{4\div2}=\frac{1}{2} \]

Similarly, we have:
\[ \frac{120}{20}=\frac{120\cdot10}{20\cdot10}=\frac{1200}{200} \]
\[ \frac{120}{20}=\frac{120\div10}{20\div10}=\frac{12}{2} \]

Since the fraction line is the same as the division symbol, we can rewrite our observation as:
\[ 120\div20=1200\div200 \]
\[ 120\div20=12\div2 \]

The rule is: \textbf{In a division, if we move the decimal point in both numbers to the same direction for the same number of times, the quotient doesn't change.} For example:
\[
\begin{aligned}[t]
	1.23 \div 0.3 &= 12.3 \div 3 &&\text{Decimal point in both numbers moved to the right once.} \\
	1.23 \div 0.3 &= 123 \div 30 &&\text{Decimal point in both numbers moved to the right twice.} \\
	1.23 \div 0.3 &= 0.123 \div 0.03 &&\text{Decimal point in both numbers moved to the left once.}
\end{aligned}
\]

We will learn how to use this rule in the following examples.

\begin{myexample}
Calculate $20\div0.4$
\end{myexample}
\begin{solution}
\begin{itemize}
\item We need to change $0.4$ to an integer by moving the decimal point to the right once.
\item By the rule we learned above, we need to move the decimal point of both $20$ and $0.4$ to the right once, and we have $20\div0.4=200\div4$
\item Now the division is easy. We have: $20\div0.4=200\div4=50$
\end{itemize}
\end{solution}

\begin{myexample}
Calculate $0.12\div0.004$
\end{myexample}
\begin{solution}
\begin{itemize}
\item We need to change $0.004$ to an integer by moving the decimal point to the right three times.
\item By the rule we learned above, we need to move the decimal point of both $0.12$ and $0.004$ to the right three times, and we have $0.12\div0.004=120\div4$
\item Now the division is easy. We have: $0.12\div0.004=120\div4=30$
\end{itemize}
\end{solution}

